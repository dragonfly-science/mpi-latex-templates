\documentclass{mpi-plenary}
\title{The Fisheries Plenary}{The Fisheries Plenary}
\subtitle{}

\author{The Ministry for Primary Industries}

\usepackage{amsmath}
\usepackage{longtable}
\usepackage{placeins}
\usepackage{multirow}
\usepackage{dcolumn}
\usepackage[font={small,bf},captionskip=0pt,
            nearskip=0pt,farskip=0pt,position=top,
            justification=justified,singlelinecheck=false]{subfig}

\newcommand{\degrees}{\ensuremath{^\circ}}
\newcommand{\minutes}{\ensuremath{^\prime}}

\makeatletter
\setlength{\@fptop}{0pt}
\setlength{\@fpbot}{0pt plus 1fil}
\makeatother

% Added by me
\newcounter{chapter}

\providecommand{\tightlist}{%
  \setlength{\itemsep}{0pt}\setlength{\parskip}{0pt}}

% end

\usepackage{amsthm}
\newtheorem{theorem}{Theorem}[chapter]
\newtheorem{lemma}{Lemma}[chapter]
\theoremstyle{definition}
\newtheorem{definition}{Definition}[chapter]
\newtheorem{corollary}{Corollary}[chapter]
\newtheorem{proposition}{Proposition}[chapter]
\theoremstyle{definition}
\newtheorem{example}{Example}[chapter]
\theoremstyle{definition}
\newtheorem{exercise}{Exercise}[chapter]
\theoremstyle{remark}
\newtheorem*{remark}{Remark}
\newtheorem*{solution}{Solution}
\begin{document}
\maketitle

%\cleardoublepage\newpage\thispagestyle{empty}\null
%\cleardoublepage\newpage\thispagestyle{empty}\null
%\cleardoublepage\newpage
\thispagestyle{empty}
%\begin{center}
%\includegraphics{images/dedication.pdf}
%\end{center}

\setlength{\abovedisplayskip}{-5pt}
\setlength{\abovedisplayshortskip}{-5pt}

{
\hypersetup{linkcolor=}
\setcounter{tocdepth}{2}
\tableofcontents
}
\listoftables
\listoffigures
\chapter{Introduction to the Fisheries
Plenary}\label{introduction-to-the-fisheries-plenary}

\section{Introduction}\label{introduction}

This report summarises the conclusions and recommendations from the
meetings of the Fisheries Assessment Working Groups and the Fisheries
Assessment Plenary held since last year's Plenary report was published.
The meetings were convened to assess the fisheries managed within the
Quota Management System, as well as other important fisheries in the New
Zealand EEZ, and to discuss various matters that pertain to fisheries
assessments.

In addition, summaries of environmental effects of fishing from research
presented to the Aquatic Environment Working Group (AEWG) that have
relevance to fishery management have been incorporated for selected
species. Paragraph 11 (page 15) of the Terms of Reference for Fisheries
Assessment Working Groups (FAWGs) includes ``\ldots{}information and
advice on other management considerations (e.g., \ldots{}by-catch
issues, effects of fishing on habitat\ldots{})'', and states that
``Sections of the Working Group reports related to bycatch and other
environmental effects of fishing will be reviewed by the Aquatic
Environment Working Group although the relevant FAWG is encouraged to
identify to the AEWG Chair any major discrepancies between these
sections and their understanding of the operation of relevant
fisheries''. In addition, the Terms of Reference for the AEWG (Paragraph
8, page 18) specifies the need ``to review and revise existing
environmental and ecosystem consideration sections of Fisheries
Assessment Plenary report text based on new data or analyses, or other
relevant information''.

The report addresses, for each species, relevant aspects of the
Fisheries Act 1996 and related considerations, as defined in the Terms
of Reference for Fisheries Assessment Working Groups for 2018. In all
cases, consideration has been based on and limited by the best available
information. The purpose has been to provide objective, independent
assessments of the current status of the fish stocks.

There are two types of catch limits used in this document -- total
allowable catch (TAC) and total allowable commercial catch (TACC). The
current definition is that a TAC is a limit on the total removals from
the stock, including those taken by the commercial, recreational and
customary non-commercial sectors, illegal removals and all other
mortality to a stock caused by fishing. A TACC is a limit on the catch
taken by the commercial sector only. The definition of TAC was changed
in the 1990 Fisheries Amendment Act when the term TACC was introduced.
Before 1990, the term TAC applied only to commercial fishing. In the
Landings and TAC tables in this report, the TAC figures equate to the
TACC unless otherwise specified.

Only actual TACCs are provided. The actual TACCs are the values as of
the last day of the fishing year; e.g., 30 September.

In considering customary non-commercial, and recreational interests, the
focus has been on current interests and activities rather than
historical activities. In most cases, there is little information
available on the nature and extent of non-commercial interests, although
estimates of recreational harvest are available in some instances.
Information on illegal catches and other sources of mortality is
provided where available.

\subsection{Yield Benchmarks}\label{yield-benchmarks}

The biological reference points, Maximum Constant Yield (MCY) and
Current Annual Yield (CAY) first used in the 1988 assessment continue to
be used in some stock assessments. This approach is described in the
section of this report titled ``Guide to Biological Reference Points for
Fisheries Assessment Meetings''.

\subsection{Sources of Data}\label{sources-of-data}

A major source of information for these assessments is the fisheries
statistics system. It is important to maintain and develop this system
to provide adequate and timely data for stock assessments.

\subsection{Other Information}\label{other-information}

For some assessments, draft Fisheries Assessment Reports that more fully
describe the data and the analyses have been prepared in time for the
Working Group or Plenary process. Once finalised, these documents are
placed on the Ministry's Fisheries website in a searchable database.

\subsection{Environmental Effects of
Fishing}\label{environmental-effects-of-fishing}

The scientific information to assess the environmental effects of
fishing and enable this outcome comes primarily from research
commissioned by the Ministry and, for protected species only, the
Department of Conservation (DOC). The work is reviewed by the Aquatic
Environment Working Group (AEWG) (or a similar DOC technical working
group) or by the Biodiversity Research Advisory Group (BRAG). The
Ministry developed an ``Aquatic Environment and Biodiversity Annual
Review'', which summarises the current state of knowledge on the
environmental interactions between fisheries and the aquatic
environment. The Aquatic Environment and Biodiversity Annual Review
assesses the various known and potential effects of fishing on an
issue-by-issue basis (e.g., the total impact of all bottom trawl and
dredge fisheries on benthic habitat), whereas relatively brief
fishery-specific summaries have been progressively included in this
report since 2005, starting with hoki. These fishery-specific sections
are reviewed by AEWG rather than by the FAWGs responsible for the stock
assessment sections in each Working Group report.

\subsection{Status of Stocks Summary
Tables}\label{status-of-stocks-summary-tables}

Since 2009, the key information relevant to providing more comprehensive
and meaningful information for fisheries managers, stakeholders and
other interested parties has been summarised at the end of each chapter
in a table format using the Guidelines for Status of the Stocks Summary
Tables on pages 39--44. Beginning in 2012, selected Status of Stocks
tables have incorporated a new science information quality ranking
system, as specified in the Research and Science Information Standard
for New Zealand Fisheries (2011). Beginning in 2013, selected Status of
Stocks tables have incorporated explicit statements regarding the status
of fisheries relative to overfishing thresholds.

\subsection{Glossary of Common Technical
Terms}\label{glossary-of-common-technical-terms}

\protect\hypertarget{def-abundance-index}{}{} \textbf{Abundance Index}:
A quantitative measure of fish density or abundance, usually as a
relative time series. An abundance index can be specific to an area or
to a segment of the \protect\hyperlink{def-stock}{stock} (e.g., mature
fish), or it can refer to abundance stock-wide; the index can reflect
abundance in numbers or in weight
(\protect\hyperlink{def-biomass}{biomass}).

\protect\hypertarget{def-aewg}{}{} \textbf{AEWG}: The Aquatic
Environment (Science) Working Group.

\protect\hypertarget{def-age-frequency}{}{} \textbf{Age frequency}: The
proportions of fish of different ages in the
\protect\hyperlink{def-stock}{stock}, or in the catch taken by either
the commercial fishery or research fishing. This is often estimated
based on a sample. Sometimes called an age composition.

\protect\hypertarget{def-age-length-key}{}{} \textbf{Age-length key}:
The proportion of fish of each age in each length-group in a sample of
fish.

\protect\hypertarget{def-age-structured-stock-assessment}{}{}
\textbf{Age-structured stock assessment}: An assessment that uses a
model to estimate how the numbers at age in the stock vary over time in
order to determine the past and present
\protect\hyperlink{def-stock-status}{status} of a fish
\protect\hyperlink{def-stock}{stock}.

\protect\hypertarget{def-a50}{}{} \textbf{a50}: Either the age at which
50\% of fish are mature (= AM) or 50\% are recruited to the fishery
(=AR).

\protect\hypertarget{def-aic}{}{} \textbf{AIC}: The Akaike Information
Criterion is a measure of the relative quality of a statistical model
for a given set of data. As such, AIC provides a means for model
selection; the preferred model is the one with the minimum AIC value.

\protect\hypertarget{def-am}{}{} \textbf{AM}: Age at maturity is the age
at which fish, of a given sex, are considered to be reproductively
mature. See \protect\hyperlink{def-a50}{a50}.

\protect\hypertarget{def-amp}{}{} \textbf{AMP}: Adaptive Management
Programme. This involves increased \protect\hyperlink{def-tacc}{TACC's}
(for a limited period, usually 5 years) in exchange for which the
industry is required to provide data that will improve understanding of
\protect\hyperlink{def-stock-status}{stock status}. The industry is also
required to collect additional information (biological data and detailed
catch and effort) and perform the analyses (e.g.
\protect\hyperlink{def-cpue}{CPUE} standardisation or age structure)
necessary for monitoring the \protect\hyperlink{def-stock}{stock}.

\protect\hypertarget{def-antwg}{}{} \textbf{ANTWG}: Antarctic (Science)
Working Group.

\protect\hypertarget{def-age-of-recruitment}{}{} \textbf{AR: Age of
recruitment} is the age when fish are considered to be
\textbf{recruited} to the fishery. In \textbf{stock assessments}, this
is usually the youngest age group considered in the analyses. See
\protect\hyperlink{def-a50}{a50}.

\protect\hypertarget{def-a-to-95}{}{} \textbf{ato95}: The number of ages
between the age at which 50\% of a stock is mature (or recruited) and
the age at which 95\% of the stock is mature (or recruited).

\protect\hypertarget{def-bo}{}{} \textbf{Bo: Virgin biomass, unfished
biomass.} This is the theoretical \textbf{carrying capacity} of the
\textbf{recruited} or \textbf{vulnerable} or \textbf{spawning biomass}
of a fish \protect\hyperlink{def-stock}{stock}. In some cases, it refers
to the average biomass of the stock in the years before fishing started.
More generally, it is the average over recent years of the biomass that
theoretically would have occurred if the stock had never been fished. B0
is often estimated from stock modelling and various percentages of it
(e.g.~40\% B0) are used as \textbf{biological reference points (BRPs)}
to assess the relative status of a \protect\hyperlink{def-stock}{stock}.

\protect\hypertarget{def-bav}{}{} \textbf{Bav}: The average historical
\protect\hyperlink{def-recruited-biomass}{recruited biomass}.

\protect\hypertarget{def-bayesian-stock-assessment}{}{} \textbf{Bayesian
stock assessment}: an approach to stock assessment that provides
estimates of uncertainty (\protect\hyperlink{def-posterior}{posterior
distributions}) of the quantities of interest in the assessment. The
method allows the initial uncertainty (that before the data are
considered) to be described in the form of \textbf{priors}. If the data
are informative, they will determine the posterior distributions; if
they are uninformative, the posteriors will resemble the
\textbf{priors}. The initial model runs are called \textbf{MPD} (mode of
the posterior distribution) runs, and provide point estimates only, with
no uncertainty. Final runs (Markov Chain Monte Carlo runs or
\textbf{MCMCs}), which are often very time consuming, provide both point
estimates and estimates of uncertainty.

\protect\hypertarget{def-bbeg}{}{} \textbf{Bbeg}: The estimated
\textbf{stock biomass} at the beginning of the fishing year.

\protect\hypertarget{def-bcurrent}{}{} \textbf{BCURRENT}: Current
\protect\hyperlink{def-biomass}{biomass} in the year of the assessment
(usually a mid-year biomass).

\protect\hypertarget{def-benthic}{}{} \textbf{Benthic}: the ecological
region at the lowest level of a body of water, including the sediment
surface and some sub-surface layers

\protect\hypertarget{def-brp}{}{} \textbf{Biological Reference Point
(BRP)}: A benchmark against which the biomass or abundance of the stock,
or the fishing mortality rate (or exploitation rate), or catch itself
can be measured in order to determine stock status. These reference
points can be targets, thresholds or limits depending on their intended
use.

\protect\hypertarget{def-biomass}{}{} \textbf{Biomass}: Biomass refers
to the size of the stock in units of weight. Often, biomass refers to
only one part of the stock (e.g., spawning biomass, vulnerable biomass
or recruited biomass, the latter two of which are essentially
equivalent).

\protect\hypertarget{def-bmsy}{}{} \textbf{BMSY}: The average stock
biomass that results from taking an average catch of MSY under various
types of harvest strategies. Often expressed in terms of spawning
biomass, but may also be expressed as recruited or vulnerable biomass.

\protect\hypertarget{def-bootstrap}{}{} \textbf{Bootstrap}: A
statistical methodology used to quantify the uncertainty associated with
estimates obtained from a model. The bootstrap is often based on Monte
Carlo re-sampling of residuals from the initial model fit.

\protect\hypertarget{def-brag}{}{} \textbf{BRAG}: Biodiversity Research
Advisory Group

\protect\hypertarget{def-bref}{}{} \textbf{BREF}: A reference average
biomass usually treated as a management target.

\protect\hypertarget{def-bycatch}{}{} \textbf{Bycatch}: Refers to fish
species, or size classes of those species, caught in association with
key target species.

\protect\hypertarget{def-byear}{}{} \textbf{Byear}: Estimated or
predicted biomass in the named year (usually a mid-year biomass).

\protect\hypertarget{def-carrying-capacity}{}{} \textbf{Carrying
capacity}: The average stock size expected in the absence of fishing.
Even without fishing the stock size varies through time in response to
stochastic environmental conditions. See Bo: virgin biomass.

\protect\hypertarget{def-catch}{}{} \textbf{Catch (C)}: The total weight
(or sometimes number) of fish caught by fishing operations.

\protect\hypertarget{def-cay}{}{} \textbf{CAY}: Current annual yield is
the one year catch calculated by applying a reference fishing mortality,
Fref, to an estimate of the fishable biomass at the beginning of the
fishing year. Also see MAY.

\protect\hypertarget{def-celr}{}{} \textbf{CELR}: Catch-Effort Landing
Return.

\protect\hypertarget{def-clr}{}{} \textbf{CLR}: Catch Landing Return.

\protect\hypertarget{def-cohort}{}{} \textbf{Cohort}: Those individuals
of a stock born in the same spawning season. For annual spawners, a
year's recruitment of new individuals to a stock is a single cohort or
year-class.

\protect\hypertarget{def-collapsed}{}{} \textbf{Collapsed}: Stocks that
are below the hard limit are deemed to be collapsed.

\protect\hypertarget{def-convergence}{}{} \textbf{Convergence}: In
reference to MCMC results from a Bayesian stock assessment, convergence
means that the average and the variability of the parameter estimates
are not changing as the MCMC chain gets longer.

\protect\hypertarget{def-cpue}{}{} \textbf{CPUE}: Catch per unit effort
is the quantity of fish caught with one standard unit of fishing effort;
e.g., the number of fish taken per 1000 hooks per day or the weight of
fish taken per hour of trawling. CPUE is often assumed to be a relative
abundance index.

\protect\hypertarget{def-customary-catch}{}{} \textbf{Customary catch}:
Catch taken by tangata whenua to meet their customary needs.

\protect\hypertarget{def-cv}{}{} \textbf{CV: Coefficient of variation.}
A statistic commonly used to represent variability or uncertainty. For
example, if a biomass estimate has a CV of 0.2 (or 20\%), this means
that the error in this estimate (the difference between the estimate and
the true biomass) will typically be about 20\% of the estimate.

\protect\hypertarget{def-density-dependence}{}{}
\textbf{Density-dependence}: Fish populations are thought to
self-regulate as population biomass increases, growth may slow down,
mortality may increase, recruitment may decrease or maturity may occur
later. Growth is density-dependent if it slows down as biomass
increases.

\protect\hypertarget{def-depleted}{}{} \textbf{Depleted}: Stocks that
are below the soft limit are deemed to be depleted. Stocks can become
depleted through overfishing, or environmental factors, or a combination
of the two.

\protect\hypertarget{def-discards}{}{} \textbf{Discards}: The portion of
the catch thrown away at sea

\protect\hypertarget{def-dwwg}{}{} \textbf{DWWG}: The Deepwater
(Science) Working Group.

\protect\hypertarget{def-ecer}{}{} \textbf{ECER}: Eel Catch-Effort
Return.

\protect\hypertarget{def-eclr}{}{} \textbf{ECLR}: Eel Catch Landing
Return.

\protect\hypertarget{def-ecosystem}{}{} \textbf{Ecosystem}: a biological
community of interacting organisms and their physical environment.

\protect\hypertarget{def-eez}{}{} \textbf{EEZ}: An Exclusive Economic
Zone is a maritime zone beyond the Territorial Sea over which the
coastal state has sovereign rights over the exploration and use of
marine resources. Usually, a state's EEZ extends to a distance of 200
nautical miles (370 km) out from its coast, except where resulting
points would be closer to another country.

\protect\hypertarget{def-equilibrium}{}{} \textbf{Equilibrium}: A
theoretical model state that arises when the fishing mortality,
exploitation pattern and other fishery or stock characteristics (growth,
natural mortality, recruitment) do not change from year to year.

\protect\hypertarget{def-exploitable-biomass}{}{} \textbf{Exploitable
biomass}: Refers to that portion of a stock's biomass that is available
to the fishery. Also called recruited biomass or vulnerable biomass.

\protect\hypertarget{def-exploitation-pattern}{}{} \textbf{Exploitation
pattern}: The relative proportion of each age or size class of a stock
that is vulnerable to fishing. See selectivity ogive.

\protect\hypertarget{def-exploitation-rate}{}{} \textbf{Exploitation
rate}: The proportion of the recruited or vulnerable biomass that is
caught during a certain period, usually a fishing year.

\protect\hypertarget{def-f}{}{} \textbf{F}: The fishing intensity or
fishing mortality rate is that part of the total mortality rate applying
to a fish stock that is caused by fishing. Usually expressed as an
instantaneous rate.

\protect\hypertarget{def-f0.1}{}{} \textbf{F0.1}: The fishing mortality
rate at which the increase in equilibrium yield per recruit in weight
per unit of effort is 10\% of the yield per recruit produced by the
first unit of effort on the unexploited stock (i.e., the slope of the
yield per recruit curve for the F0.1 rate is only 1/10th of the slope of
the yield per recruit curve at its origin).

\protect\hypertarget{def-f40b0}{}{} \textbf{F40\%B0}: The fishing
mortality rate associated with a biomass of 40\% B0 at equilibrium or on
average.

\protect\hypertarget{def-f40spr}{}{} \textbf{F40\%SPR}: The fishing
mortality rate associated with a spawning biomass per recruit (SPR) (or
equivalently a spawning potential ratio) of 40\% B0 at equilibrium or on
average.

\protect\hypertarget{def-fawgs}{}{} \textbf{FAWGs}: Fisheries Assessment
(Science) Working Groups.

\protect\hypertarget{def-fishing-intensity}{}{} \textbf{Fishing
intensity}: A general term that encompasses the related concepts of
fishing mortality and exploitation rate.

\protect\hypertarget{def-fishing-mortality}{}{} \textbf{Fishing
mortality}: That part of the total mortality rate applying to a fish
stock that is caused by fishing. Usually expressed as an instantaneous
rate.

\protect\hypertarget{def-fishing-year}{}{} \textbf{Fishing year}: For
most fish stocks, the fishing year runs from 1 October in one year to 30
September in the next. The second year is often used as shorthand for
the split years. For example, 2015 is shorthand for 2014--15.

\protect\hypertarget{def-fma}{}{} \textbf{FMA}: Fishery Management Area.
The New Zealand EEZ is divided into 10 fisheries management units:

\protect\hypertarget{def-fmax}{}{} \textbf{Fmax}: The fishing mortality
rate that maximises equilibrium yield per recruit. Fmax is the fishing
mortality level that defines growth overfishing. In general, Fmax is
different from FMSY (the fishing mortality that maximises sustainable
yield), and is always greater than or equal to FMSY, depending on the
stock-recruitment relationship.

\protect\hypertarget{def-fmey}{}{} \textbf{FMEY}: The fishing mortality
corresponding to the maximum (sustainable) economic yield.

\protect\hypertarget{def-fmsy}{}{} \textbf{FMSY}: The fishing mortality
rate that, if applied constantly, would result in an average catch
corresponding to the Maximum Sustainable Yield (MSY) and an average
biomass corresponding to BMSY. Usually expressed as an instantaneous
rate.

\protect\hypertarget{def-fref}{}{} \textbf{Fref}: The fishing mortality
that is associated with an average biomass of BREF.

\protect\hypertarget{def-frml}{}{} \textbf{FRML}: Fisheries Related
Mortality Limit.

\protect\hypertarget{def-growth-overfishing}{}{} \textbf{Growth
overfishing}: Growth overfishing occurs when the fishing mortality rate
is above Fmax. This means that on average fish are caught before they
have a chance to reach their maximum growth potential.

\protect\hypertarget{def-hard-limit}{}{} \textbf{Hard Limit}: A biomass
limit below which fisheries should be considered for closure.

\protect\hypertarget{def-harvest-strategy}{}{} \textbf{Harvest
Strategy}: For the purpose of the Harvest Strategy Standard, a harvest
strategy simply specifies target and limit reference points and
management actions associated with achieving the targets and avoiding
the limits.

\protect\hypertarget{def-hms}{}{} \textbf{HMS}: Highly Migratory
Species.

\protect\hypertarget{def-hmswg}{}{} \textbf{HMSWG}: Highly Migratory
Species (Science) Working Group.

\protect\hypertarget{def-hyperdepletion}{}{} \textbf{Hyperdepletion}:
The situation where an abundance index, such as CPUE, decreases faster
than the true abundance.

\protect\hypertarget{def-hyperstability}{}{} \textbf{Hyperstability}:
The situation where an abundance index, such as CPUE, decreases more
slowly than the true abundance.

\protect\hypertarget{def-incidental-capture}{}{} \textbf{Incidental
capture}: Refers to non-fish and protected species which were not
targeted, but were caught.

\protect\hypertarget{def-index}{}{} \textbf{Index}: Same as an abundance
index.

\protect\hypertarget{def-lcer}{}{} \textbf{LCER}: Longline Catch-Effort
Return.

\protect\hypertarget{def-length-frequency}{}{} \textbf{Length
frequency}: The distribution of numbers at length from a sample of the
catch taken by either the commercial fishery or research fishing. This
is sometimes called a length composition.

\protect\hypertarget{def-length-structured-stock-assessment}{}{}
\textbf{Length-Structured Stock Assessment}: An assessment that uses a
model to estimate how the numbers at length in the stock vary over time
in order to determine the past and present status of a fish stock.

\protect\hypertarget{def-limit}{}{} \textbf{Limit}: a biomass or fishing
mortality reference point that should be avoided with high probability.
The Harvest Strategy Standard defines both soft limits and hard limits.

\protect\hypertarget{def-m}{}{} \textbf{M}: The (instantaneous) natural
mortality rate is that part of the total mortality rate applying to a
fish stock that is caused by predation and other natural events.

\protect\hypertarget{def-mafwg}{}{} \textbf{MAFWG}: Marine Amateur
Fisheries (Science) Working Group.

\protect\hypertarget{def-malfirm}{}{} \textbf{MALFIRM}: Maximum
Allowable Limit of Fishing Related Mortality.

\protect\hypertarget{def-maturity}{}{} \textbf{Maturity}: Refers to the
ability of fish to reproduce.

\protect\hypertarget{def-maturity-ogive}{}{} \textbf{Maturity ogive}: A
curve describing the proportion of fish of different ages or sizes that
are mature.

\protect\hypertarget{def-may}{}{} \textbf{MAY}: Maximum average yield is
the average maximum sustainable yield that can be produced over the long
term under a constant fishing mortality strategy, with little risk of
stock collapse.~ A constant fishing mortality strategy means catching a
constant percentage of the biomass present at the beginning of each
fishing year. MAY is the long-term average annual catch when the catch
each year is the CAY. Also see CAY.

\protect\hypertarget{def-mcmc}{}{} \textbf{MCMC}: Markov Chain Monte
Carlo. See Bayesian stock assessment.

\protect\hypertarget{def-mcy}{}{} \textbf{MCY}: Maximum constant yield
is the maximum sustainable yield that can be produced over the long term
by taking the same catch year after year, with little risk of stock
collapse.

\protect\hypertarget{def-midwg}{}{} \textbf{MIDWG}: Middle-depths
(Science) Working Group.

\protect\hypertarget{def-mid-year-biomass}{}{} \textbf{Mid-year
biomass}: The biomass after half the year's catch has been taken.

\protect\hypertarget{def-mls}{}{} \textbf{MLS}: Minimum Legal Size. Fish
above the MLS can be retained while those below it must be returned to
the sea.

\protect\hypertarget{def-model}{}{} \textbf{Model}: A set of equations
that represents the population dynamics of a fish stock.

\protect\hypertarget{def-monte-carlo-simulation}{}{} \textbf{Monte Carlo
Simulation}: is an approach whereby the inputs that are used for a
calculation are re-sampled many times assuming that the inputs follow
known statistical distributions. The Monte Carlo method is used in many
applications such as Bayesian stock assessments, parametric bootstraps
and stochastic projections.

\protect\hypertarget{def-mpd}{}{} \textbf{MPD}: Mode of the (joint)
posterior distribution. See Bayesian stock assessment.

\protect\hypertarget{def-msy}{}{} \textbf{MSY}: Maximum sustainable
yield is the largest long-term average catch or yield that can be taken
from a stock under prevailing ecological and environmental conditions,
and the current selectivity patterns exhibited by the fishery.

\protect\hypertarget{def-msy-compatible-reference-points}{}{}
\textbf{MSY-compatible reference points}: MSY-compatible references
points include BMSY, FMSY and MSY itself, as well as analytical and
conceptual proxies for each of these three quantities.

\protect\hypertarget{def-natural-mortality-rate}{}{} \textbf{Natural
mortality (rate)}: That part of the total mortality rate applying to a
fish stock that is caused by predation and other natural events. Usually
expressed as an instantaneous rate.

\protect\hypertarget{def-ncelr}{}{} \textbf{NCELR}: Set Net Catch-Effort
Landing Return.

\protect\hypertarget{def-nins}{}{} \textbf{NINS}: Northern Inshore
(Science) Working Group.

\protect\hypertarget{def-objective-function}{}{} \textbf{Objective
function}: An equation to be optimised (minimised or maximised) given
certain constraints using non-linear programming techniques.

\protect\hypertarget{def-otolith}{}{} \textbf{Otolith}: One of the small
bones or particles of calcareous substance in the internal ear of
teleosts (bony fishes) that are used to determine their age.

\protect\hypertarget{def-overexploitation}{}{}
\textbf{Overexploitation}: A situation where observed exploitation (or
fishing mortality) rates are higher than target levels.

\protect\hypertarget{def-overfishing}{}{} \textbf{Overfishing}: A
situation where observed fishing mortality (or exploitation) rates are
higher than target or threshold levels.

\protect\hypertarget{def-partition}{}{} \textbf{Partition}: The way in
which a fish stock or population is characterised, or split, in a stock
assessment model; for example, by sex, age and maturity.

\protect\hypertarget{def-pcelr}{}{} \textbf{PCELR}: Paua Catch Effort
and Landing Return.

\protect\hypertarget{def-population}{}{} \textbf{Population}: A group of
fish of one species that shares common ecological and genetic features.
The stocks defined for the purposes of stock assessment and management
do not necessarily coincide with self-contained populations.

\protect\hypertarget{def-population-dynamics}{}{} \textbf{Population
dynamics}: In general, refers to the biological and fishing processes
that result in changes in fish stock abundance over time.

\protect\hypertarget{def-posterior}{}{} \textbf{Posterior}: a
mathematical description of the uncertainty in some quantity (e.g.,
biomass) estimated in a Bayesian stock assessment. This is generally
depicted as a frequency distribution (often plotted along with the prior
distribution to show how much the two diverge).

\textbf{Potential Biological Removal (PBR)} an estimate of the number of
seabirds that may be killed without causing the population to decline
below half the carrying capacity.

\protect\hypertarget{def-pre-recruit}{}{} \textbf{Pre-recruit}: An
individual that has not yet entered the fished component of the stock
(because it is either too young or too small to be vulnerable to the
fishery).

\protect\hypertarget{def-prior}{}{} \textbf{Prior}: available
information (often in the form of expert opinion) regarding the
potential range of values of a parameter in a Bayesian stock assessment.
Uninformative priors are used where there is no such information.

\protect\hypertarget{def-production-model}{}{} \textbf{Production
Model}: A stock model that describes how the stock biomass changes from
year to year (or, how biomass changes in equilibrium as a function of
fishing mortality), but which does not keep track of the age or length
frequency of the stock. The simplest production functions aggregate all
of the biological characteristics of growth, natural mortality and
reproduction into a simple, deterministic model using three or four
parameters. Production models are primarily used in simple data
situations, where total catch and effort data are available but
age-structured information is either unavailable or deemed to be less
reliable (although some versions of production models allow the use of
age-structured data).

\protect\hypertarget{def-productivity}{}{} \textbf{Productivity}:
Productivity is a function of the biology of a species and the
environment in which it lives. It depends on growth rates, natural
mortality, age at maturity, maximum average age and other relevant life
history characteristics. Species with high productivity are able to
sustain higher rates of fishing mortality than species with lower
productivity. Generally, species with high productivity are more
resilient and take less time to rebuild from a depleted state.

\protect\hypertarget{def-projection}{}{} \textbf{Projection}:
Predictions about trends in stock size and fishery dynamics in the
future. Projections are made to address ``what-if'' questions of
relevance to management. Short-term (1--5 years) projections are
typically used in support of decision-making. Longer term projections
become much more uncertain in terms of absolute quantities, because the
results are strongly dependent on recruitment, which is very difficult
to predict. For this reason, long-term projections are more useful for
evaluating overall management strategies than for making short-term
decisions.

\protect\hypertarget{def-proxy}{}{} \textbf{Proxy}: A surrogate for
BMSY, FMSY or MSY that has been demonstrated to approximate one of these
three metrics through theoretical or empirical studies.

\protect\hypertarget{def-q}{}{} \textbf{q}: Catchability is the
proportion of fish that are caught by a defined unit of fishing effort.
The constant relating an abundance index to the true biomass (the
abundance index is approximately equal to the true biomass multiplied by
the catchability).

\protect\hypertarget{def-quota-management-areas}{}{} \textbf{Quota
Management Areas (QMA)}: QMAs are geographic areas within which fish
stocks are managed in the TS and EEZ.

\protect\hypertarget{def-quota-management-system}{}{} \textbf{Quota
Management System (QMS)}: The QMS is the name given to the system by
which the total commercial catch from all the main fish stocks found
within New Zealand's 200 nautical mile EEZ is regulated.

\protect\hypertarget{def-recruit}{}{} \textbf{Recruit}: An individual
that has entered the fished component of the stock. Fish that are not
recruited are either not catchable by the gear used (e.g., because they
are too small) or live in areas that are not fished.

\protect\hypertarget{def-recruited-biomass}{}{} \textbf{Recruited
biomass}: Refers to that portion of a stock's biomass that is available
to the fishery; also called exploitable biomass or vulnerable biomass.

\protect\hypertarget{def-recruitment}{}{} \textbf{Recruitment}: The
addition of new individuals to the fished component of a stock. This is
determined by the size and age at which fish are first caught.

\protect\hypertarget{def-reference-point}{}{} \textbf{Reference Point}:
A benchmark against which the biomass or abundance of the stock or the
fishing mortality rate (or exploitation rate) can be measured in order
to determine its status. These reference points can be targets,
thresholds or limits depending on their intended use.

\protect\hypertarget{def-rlwg}{}{} \textbf{RLWG}: Rock Lobster (Science)
Working Group.

\protect\hypertarget{def-samwg}{}{} \textbf{SAMWG}: Stock Assessment
Methods (Science) Working Group.

\protect\hypertarget{def-sav}{}{} \textbf{SAV}: The average historical
spawning biomass.

\protect\hypertarget{def-selectivity-ogive}{}{} \textbf{Selectivity
ogive}: Curve describing the relative vulnerability of fish of different
ages or sizes to the fishing gear used.

\protect\hypertarget{def-sfwg}{}{} \textbf{SFWG}: The Shellfish
(Science) Working Group.

\protect\hypertarget{def-sins}{}{} \textbf{SINS}: Southern Inshore
(Science) Working Group.

\protect\hypertarget{def-soft-limit}{}{} \textbf{Soft Limit}: A biomass
limit below which the requirement for a formal, time-constrained
rebuilding plan is triggered.

\protect\hypertarget{def-spawning-biomass}{}{} \textbf{Spawning
biomass}: The total weight of sexually mature fish in the stock. This
quantity depends on the abundance of year classes, the exploitation
pattern, the rate of growth, both fishing and natural mortality rates,
the onset of sexual maturity, and environmental conditions. Same as
mature biomass.

\protect\hypertarget{def-spawning-biomass-per-recruit-or-spawning-potential-ratio}{}{}
\textbf{Spawning (biomass) Per Recruit or Spawning Potential Ratio
(SPR)}: The expected lifetime contribution to the spawning biomass for
the average recruit to the fishery. For a given exploitation pattern,
rate of growth, maturity schedule and natural mortality, an equilibrium
value of SPR can be calculated for any level of fishing mortality. SPR
decreases monotonically with increasing fishing mortality.

\protect\hypertarget{def-statistical-area}{}{} \textbf{Statistical
area}: See the map below for the official Territorial Sea and Exclusive
Economic Zone (EEZ) statistical areas.

\protect\hypertarget{def-steepness}{}{} \textbf{Steepness}: A parameter
of stock-recruitment relationships that determines how rapidly, or
steeply, it rises from the origin, and therefore how resilient a stock
is to rebounding from a depleted state. It equates to the proportion of
virgin recruitment that corresponds to 20\% B0. A steepness value
greater than about 0.9 is considered to be high, while one less than
about 0.6 is considered to be low. The minimum value is 0.2.

\protect\hypertarget{def-stock}{}{} \textbf{Stock}: The term has
different meanings. Under the Fisheries Act, it is defined with
reference to units for the purpose of fisheries management (Fishstock).
On the other hand, a biological stock is a population of a given species
that forms a reproductive unit and spawns little if at all with other
units. However, there are many uncertainties in defining spatial and
temporal geographical boundaries for such biological units that are
compatible with established data collection systems. For this reason,
the term ``stock'' is often synonymous with an assessment / management
unit, even if there is migration or mixing of some components of the
assessment/management unit between areas.

\protect\hypertarget{def-stock-assessment}{}{} \textbf{Stock
assessment}: The analysis of available data to determine stock status,
usually through application of statistical and mathematical tools to
relevant data in order to obtain a quantitative understanding of the
status of the stock relative to defined management benchmarks or
reference points (e.g.~BMSY and/or FMSY).

\protect\hypertarget{def-stock-recruitment-relationship}{}{}
\textbf{Stock-recruitment relationship}: An equation describing how the
expected number of recruits to a stock varies as the spawning biomass
changes. The most frequently used stock-recruitment relationship is the
asymptotic Beverton-Holt equation, in which the expected number of
recruits changes very slowly at high levels of spawning biomass.

\protect\hypertarget{def-stock-status}{}{} \textbf{Stock status}: Refers
to a determination made, on the basis of stock assessment results, about
the current condition of the stock. Stock status is often expressed
relative to management benchmarks and biological reference points such
as BMSY or B0 or FMSY or F\%SPR. For example, the current biomass may be
said to be above or below BMSY or to be at some percentage of B0.
Similarly, fishing mortality may be above or below FMSY or F\%SPR.

\protect\hypertarget{def-stock-structure}{}{} \textbf{Stock structure}:
(1) Refers to the geographical boundaries of the stocks assumed for
assessment and management purposes (e.g., albacore tuna may be assumed
to be comprised of two separate stocks in the North Pacific and South
Pacific), (2) Refers to boundaries that define self-contained stocks in
a genetic sense, (3) refers to known, inferred or assumed patterns of
residence and migration for stocks that mix with one another.

\protect\hypertarget{def-surplus-production}{}{} \textbf{Surplus
production}: The amount of biomass produced by the stock (through growth
and recruitment) over and above that which is required to maintain the
{[}total stock{]} biomass at its current level. If the catch in each
year is equal to the surplus production then the biomass will not
change.

\protect\hypertarget{def-sustainability}{}{} \textbf{Sustainability}:
Pertains to the ability of a fish stock to persist in the long-term.
Because fish populations exhibit natural variability, it is not possible
to keep all fishery and stock attributes at a constant level
simultaneously, thus sustainable fishing does not imply that the fishery
and stock will persist in a constant equilibrium state. Because of
natural variability, even if FMSY could be achieved exactly each year,
catches and stock biomass will oscillate around their average MSY and
BMSY levels, respectively. In a more general sense, sustainability
refers to providing for the needs of the present generation while not
compromising the ability of future generations to meet theirs.

\protect\hypertarget{def-tac}{}{} \textbf{TAC}: Total Allowable Catch is
the sum of the Total Allowable Commercial Catch (TACC) and the
allowances for customary Maori interests, recreational fishery interests
and other sources of fishing-related mortality that can be taken in a
given period, usually a year.~

\protect\hypertarget{def-tacc}{}{} \textbf{TACC}: Total Allowable
Commercial Catch is the total regulated commercial catch from a stock in
a given time period, usually a fishing year.

\protect\hypertarget{def-target}{}{} \textbf{Target}: Generally, a
biomass, fishing mortality or exploitation rate level that management
actions are designed to achieve with at least a 50\% probability.

\protect\hypertarget{def-threshold}{}{} \textbf{Threshold}: Generally, a
biological reference point that raises a ``red flag'' indicating that
biomass has fallen below the target, or fishing mortality or
exploitation rate has increased above its target, to the extent that
additional management action may be required in order to prevent the
stock from declining further and possibly breaching the soft limit.

\protect\hypertarget{def-tcepr}{}{} \textbf{TCEPR}: Trawl Catch-Effort
Processing Return.

\protect\hypertarget{def-tcer}{}{} \textbf{TCER}: Trawl Catch-Effort
Return.

\protect\hypertarget{def-tlcer}{}{} \textbf{TLCER}: Tuna Longline
Catch-Effort Return.

\protect\hypertarget{def-ts:-territorial-sea}{}{} \textbf{TS:
Territorial Sea}: a belt of coastal waters extending at most 12 nautical
miles (22.2 km; 13.8 mi) from the baseline (usually the mean low-water
mark) of a coastal state.

\protect\hypertarget{def-umsy}{}{} \textbf{UMSY}: The exploitation rate
associated with the maximum sustainable yield.

\protect\hypertarget{def-u40b0}{}{} \textbf{U40\%B0}: The exploitation
rate associated with a biomass of 40\% B0 at equilibrium or on average.
von Bertalanffy equation: An equation describing how fish increase in
length as they grow older. The mean length (L) at age a is

where L∞ is the average length of the oldest fish, k is the average
growth rate (Brody coefficient) and t0 is a constant.

\protect\hypertarget{def-vulnerable-biomass}{}{} \textbf{Vulnerable
biomass}: Refers to that portion of a stock's biomass that is available
to the fishery. Also called exploitable biomass or recruited biomass.

\protect\hypertarget{def-year-class-cohort}{}{} \textbf{Year class
(cohort)}: Fish in a stock that were born in the same year.
Occasionally, a stock produces a very small or very large year class
which can be pivotal in determining stock abundance in later years.

\protect\hypertarget{def-yield}{}{} \textbf{Yield}: Catch expressed in
terms of weight.

\protect\hypertarget{def-yield-per-recruit-ypr}{}{} \textbf{Yield per
Recruit (YPR)}: The expected lifetime yield for the average recruit. For
a given exploitation pattern, rate of growth, and natural mortality, an
equilibrium value of YPR can be calculated for each level of fishing
mortality. YPR analyses may play an important role in advice for
management, particularly as they relate to minimum size controls.

\protect\hypertarget{def-z}{}{} \textbf{Z}: Total mortality rate. The
sum of natural and fishing mortality rates.

\section{Terms of Reference for Fisheries Assessment Working Groups
(FAWGs) in
2018}\label{terms-of-reference-for-fisheries-assessment-working-groups-fawgs-in-2018}

\subsection{Overall purpose}\label{overall-purpose}

The purpose of the FAWGs is to assess the status of fish stocks managed
within the Quota Management System, as well as other important species
of interest to New Zealand. Based on scientific information the FAWGs
assess the current status of fish stocks or species relative to
MSY-compatible reference points and other relevant indicators of stock
status, conduct projections of stock size and status under alternative
management scenarios, and review results from relevant research
projects. They do not make management recommendations or decisions (this
responsibility lies with Fisheries New Zealand (FNZ) fisheries managers
and the Minister responsible for fisheries).

\subsection{Preparatory tasks}\label{preparatory-tasks}

\begin{enumerate}
\def\labelenumi{\arabic{enumi}.}
\item
  Prior to the beginning of the main sessions of FAWG meetings (January
  to May and September to November), FNZ fisheries scientists will
  produce a list of stocks and issues for which new stock assessments or
  evaluations are likely to become available prior to the next scheduled
  sustainability rounds. This list will include stocks for which the
  fishing industry and others intend to directly purchase scientific
  analyses. It is therefore incumbent on those purchasing research to
  inform the relevant FAWG chair of their intentions at least three
  months prior to the start of the sustainability round. FAWG Chairs
  will determine the final timetables and agendas for each Working
  Group.
\item
  At least six months prior to the main sessions of FAWG meetings, FNZ
  fisheries managers will alert FNZ science managers and the Principal
  Advisor Fisheries Science to unscheduled special cases for which
  assessments or evaluations are urgently needed.
\end{enumerate}

\subsection{Technical objectives}\label{technical-objectives}

\begin{enumerate}
\def\labelenumi{\arabic{enumi}.}
\setcounter{enumi}{2}
\item
  To review new research information on stock structure, productivity,
  abundance and related topics for each fish stock/issue under the
  purview of individual FAWGs.
\item
  Where possible, to derive appropriate MSY-compatible reference points1
  for use as reference points for determining stock status, based on the
  Harvest Strategy Standard for New Zealand Fisheries2 (the Harvest
  Strategy Standard).
\item
  To conduct stock assessments or evaluations for selected fish stocks
  in order to determine the status of the stocks relative to
  MSY-compatible reference points1 and associated limits, based on the
  ``Guide to Biological Reference Points for Fisheries Assessment
  Meetings'', the Harvest Strategy Standard, and relevant management
  reference points and performance measures set by fisheries managers.
\item
  For stocks where the status is unknown, FAWGs should use existing data
  and analyses to draw logical conclusions about likely future trends in
  biomass levels and/or fishing mortality (or exploitation) rates if
  current catches and/or TACs/TACCs are maintained, or if fishers or
  fisheries managers are considering modifying them in other ways.
\item
  Where appropriate and practical, to conduct projections of likely
  future stock status using alternative fishing mortality (or
  exploitation) rates or catches and other relevant management actions,
  based on the Harvest Strategy Standard and input from the FAWG and
  fisheries managers.
\item
  For stocks that are deemed to be depleted or collapsed, to develop
  alternative rebuilding scenarios based on the Harvest Strategy
  Standard and input from the FAWG and fisheries managers.
\item
  For fish stocks for which new stock assessments or analyses are not
  conducted in the current year, to review the existing Fisheries
  Assessment Plenary report text on the ``Status of the Stocks'' in
  order to determine whether the latest reported stock status summary is
  still relevant; else to revise the evaluations of stock status based
  on new data or analyses, or other relevant information.
\end{enumerate}

\subsection{Working Group reports}\label{working-group-reports}

\begin{enumerate}
\def\labelenumi{\arabic{enumi}.}
\setcounter{enumi}{9}
\item
  To include in the Working Group report information on commercial,
  Māori customary, non-commercial and recreational interests in the
  stock; as well as all other mortality to that stock caused by fishing,
  which might need to be allowed for in setting a TAC or TACC. Estimates
  of recreational harvest will normally be provided by the Marine
  Amateur Fisheries Working Group (MAFWG).
\item
  To provide information and advice on other management considerations
  (e.g.~area boundaries, by-catch issues, effects of fishing on habitat,
  other sources of mortality, and input controls such as mesh sizes and
  minimum legal sizes) required for specifying sustainability measures.
  Sections of the Working Group reports related to bycatch and other
  environmental effects of fishing will be reviewed by the Aquatic
  Environment Working Group (AEWG) although the relevant FAWG is
  encouraged to identify to the AEWG Chair any major discrepancies
  between these sections and their understanding of the operation of
  relevant fisheries.
\item
  To summarise the stock assessment methods and results, along with
  estimates of MSY-compatible references points and other metrics that
  may be used as benchmarks for assessing stock status.
\item
  To review, and update if necessary, the ``Status of the Stocks''
  tables in the Fisheries Assessment Plenary report for all stocks under
  the purview of individual FAWGs (including those for which a full
  assessment has not been conducted in the current year) based on new
  data or analyses, or other relevant information.
\item
  For all important stocks, to complete (and/or update) the Status of
  Stocks tables using the template provided in the Introductory chapter
  of the most recent May and November Plenary reports.
\item
  It is desirable that full agreement amongst technical experts is
  achieved on the text of the FAWG reports, particularly the ``Status of
  the Stocks'' sections, noting that the AEWG will review sections on
  bycatch and other environmental effects of fishing, and the MAFWG will
  provide text on recreational harvests. If full agreement amongst
  technical experts cannot be reached, the Chair will determine how this
  will be depicted in the FAWG report, will document the extent to which
  agreement or consensus was achieved, and record and attribute any
  residual disagreement in the meeting notes.
\end{enumerate}

\subsection{Working Group input to the
Plenary}\label{working-group-input-to-the-plenary}

\begin{enumerate}
\def\labelenumi{\arabic{enumi}.}
\setcounter{enumi}{15}
\tightlist
\item
  To advise the Principal Advisor Fisheries Science about stocks
  requiring review by the Fisheries Assessment Plenary and those stocks
  that are not believed to warrant review by the Plenary. The general
  criteria for determining which stocks should be discussed by the
  Plenary are that (i) the assessment is controversial and Working Group
  members have had difficulty reaching consensus on one or more base
  cases, or (ii) the assessment is the first for a particular stock or
  the methodology has been substantially altered since the last
  assessment, or (iii) new data or analyses have become available that
  alter the previous assessment, particularly assessments of recent or
  current stock status, or projections of likely future stock status.
  Such information could include: • new or revised estimates of
  MSY-compatible reference points, recent or current biomass,
  productivity or yield projections; • the development of a major trend
  in the catch or catch per unit effort; or • any new studies or data
  that extend understanding of stock structure, fishing patterns, or
  non-commercial activities, and result in a substantial effect on
  assessments of stock status.
\end{enumerate}

\subsection{Membership and Protocols for all Science Working
Groups}\label{membership-and-protocols-for-all-science-working-groups}

\begin{enumerate}
\def\labelenumi{\arabic{enumi}.}
\setcounter{enumi}{16}
\tightlist
\item
  FAWG members are bound by the Membership and Protocols required for
  all Science Working Group members.
\end{enumerate}

\chapter{Pāua (PAU)}\label{paua-pau}

\section{Introduction}\label{introduction-1}

\begin{center}\includegraphics{bookdown_files/figure-latex/intro-pic-1} \end{center}

Specific Working Group reports are given separately for PAU 2, PAU 3,
PAU 4, PAU 5A, PAU 5B, PAU 5D and PAU 7. The TACC for PAU 1, PAU 6 and
PAU 10 is 1.93 t, 1 t and 1 t respectively. Commercial landings for PAU
10 since 1983 have been 0 t.

\subsection{Commercial fisheries}\label{commercial-fisheries}

The commercial fishery for paua dates from the mid-1940s. In the early
years of this commercial fishery the meat was generally discarded and
only the shell was marketed, however by the late 1950s both meat and
shell were being sold. Since the 1986--87 fishing season, the eight
Quota Management Areas have been managed with an individual transferable
quota system and a total allowable catch (TAC) that is made up of total
allowed commercial catch (TACC), recreational and customary catch and
other sources of mortality.

Fishers gather paua by hand while free diving (use of underwater
breathing apparatus is not permitted). Most of the catch is from the
Wairarapa coast southwards: the major fishing areas are in the South
Island, Marlborough (PAU 7), Stewart Island (PAU 5A, 5B and 5D) and the
Chatham Islands (PAU 4). Virtually the entire commercial fishery is for
the black-foot paua, Haliotis iris, with a minimum legal size for
harvesting of 125 mm shell length. The yellow-foot paua, H. australis is
less abundant than H. iris and is caught only in small quantities; it
has a minimum legal size of 80 mm. Catch statistics include both H. iris
and H.~australis.

Up until the 2002 fishing year, catch was reported by general
statistical areas, however from 2002 onwards, a more finely scaled
system of paua specific statistical areas were put in place throughout
each QMA (refer to the QMA specific Working Group reports). Figure 1
shows the historical landings for the main PAU stocks. On 1 October 1995
PAU 5 was divided into three separate QMAs: PAU 5A, PAU 5B and PAU 5D.

\begin{figure}

{\centering \includegraphics{bookdown_files/figure-latex/weight-by-fishing-year-1-1} 

}

\caption{Historic landings for the major paua QMAs from 1983–84 to 1995–96}\label{fig:weight-by-fishing-year-1}
\end{figure}

\begin{figure}

{\centering \includegraphics{bookdown_files/figure-latex/weight-by-fishing-year-2-1} 

}

\caption{Historic landings for the major paua QMAs from 1996–97 to 2016-17}\label{fig:weight-by-fishing-year-2}
\end{figure}

Landings for PAU 1, PAU 6, PAU 10 and PAU 5 (prior to 1995) are shown in
Table \ref{tab:reported-landings}. For information on landings specific
to other paua QMAs refer to the specific Working Group reports.

\begin{tabular}{l|r|l|l|l|l|l|r|l}
\hline
PAU10Fishstock & Landings & TACC & Landings & TACC & Landings & TACC & Landings & TACC\\
\hline
1983–84* & 1.00 & - & 550 & - & 0 & - & 0 & -\\
\hline
1984–85* & 0.00 & - & 353 & - & 3 & - & 0 & -\\
\hline
1985–86* & 0.00 & - & 228 & - & 0 & - & 0 & -\\
\hline
1986–87* & 0.01 & 1 & 418.9 & 445 & 0 & 1 & 0 & 1\\
\hline
1987–88* & 0.98 & 1 & 465 & 448.98 & 0 & 1 & 0 & 1\\
\hline
1988–89* & 0.05 & 1.93 & 427.97 & 449.64 & 0 & 1 & 0 & 1\\
\hline
1989–90 & 0.28 & 1.93 & 459.46 & 459.48 & 0 & 1 & 0 & 1\\
\hline
1990–91 & 0.16 & 1.93 & 528.16 & 484.94 & 0.23 & 1 & 0 & 1\\
\hline
1991–92 & 0.27 & 1.93 & 486.76 & 492.06 & 0 & 1 & 0 & 1\\
\hline
1992–93 & 1.37 & 1.93 & 440.15 & 442.85 & 0.88 & 1 & 0 & 1\\
\hline
1993–94 & 1.05 & 1.93 & 440.39 & 442.85 & 0.1 & 1 & 0 & 1\\
\hline
1994–95 & 0.26 & 1.93 & 436.13 & 442.85 & 18.21H & 1 & 0 & 1\\
\hline
1995–96 & 0.99 & 1.93 & - & - & 28.62H & 1 & 0 & 1\\
\hline
1996–97 & 1.28 & 1.93 & - & - & 0.11 & 1 & 0 & 1\\
\hline
1997–98 & 1.28 & 1.93 & - & - & 0 & 1 & 0 & 1\\
\hline
1998–99 & 1.13 & 1.93 & - & - & 0 & 1 & 0 & 1\\
\hline
1999–00 & 0.69 & 1.93 & - & - & 1.04 & 1 & 0 & 1\\
\hline
2000–01 & 1.00 & 1.93 & - & - & 0 & 1 & 0 & 1\\
\hline
2001–02 & 0.32 & 1.93 & - & - & 0 & 1 & 0 & 1\\
\hline
2002–03 & 0.00 & 1.93 & - & - & 0 & 1 & 0 & 1\\
\hline
2003–04 & 0.05 & 1.93 & - & - & 0 & 1 & 0 & 1\\
\hline
2004–05 & 0.27 & 1.93 & - & - & 0 & 1 & 0 & 1\\
\hline
2005–06 & 0.45 & 1.93 & - & - & 0 & 1 & 0 & 1\\
\hline
2006–07 & 0.76 & 1.93 & - & - & 1 & 1 & 0 & 1\\
\hline
2007–08 & 1.14 & 1.93 & - & - & 1 & 1 & 0 & 1\\
\hline
2008–09 & 0.47 & 1.93 & - & - & 1 & 1 & 0 & 1\\
\hline
2009–10 & 0.20 & 1.93 & - & - & 1 & 1 & 0 & 1\\
\hline
2010–11 & 0.12 & 1.93 & - & - & 1 & 1 & 0 & 1\\
\hline
2011–12 & 0.77 & 1.93 & - & - & 1 & 1 & 0 & 1\\
\hline
2012–13 & 1.06 & 1.93 & - & - & 1 & 1 & 0 & 1\\
\hline
2013–14 & 0.71 & 1.93 & - & - & 1 & 1 & 0 & 1\\
\hline
2014–15 & 0.47 & 1.93 & - & - & 1 & 1 & 0 & 1\\
\hline
2015–16 & 0.13 & 1.93 & - & - & 0.84 & 1 & 0 & 1\\
\hline
2016-17 & 0.25 & 1.93 & - & - & 1.06 & 1 & 0 & 1\\
\hline
\end{tabular}

H experimental landings

* FSU data

\subsection{Recreational Fisheries}\label{recreational-fisheries}

There is a large recreational fishery for paua. Estimated catches from
telephone and diary surveys of recreational fishers (Teirney et al 1997,
Bradford 1998, Boyd \& Reilly 2004, Boyd et al 2004) are shown in Table
2a.

\begin{tabular}{l|l|l|l|l|l|l|l|l|l}
\hline
Fishstock & PAU 1 & PAU 2 & PAU 3 & PAU 5 & PAU5A & PAU5B & PAU 5D & PAU 6 & PAU 7\\
\hline
1991–92 & - & - & 35–60 & 50–80 & - & - & - & - & -\\
\hline
1992–93 & - & 37–89 & - & - & - & - & - & 0–1 & 2–7\\
\hline
1993–94 & 29–32 & - & - & - & - & - & - & - & -\\
\hline
1995–96 & 10–20 & 45–65 & - & 20–35 & - & - & - & - & -\\
\hline
1996–97 & - & - & - & N/A & - & - & 22.5 & - & -\\
\hline
1999–00 & 40–78 & 224–606 & 26–46 & 36–70 & - & - & 26–50 & 2–14 & 8–23\\
\hline
2000–01 & 16–37 & 152–248 & 31–61 & 70–121 & - & - & 43–79 & 0–3 & 4–1\\
\hline
\end{tabular}

*1991--1995 Regional telephone/diary estimates, 1995/96, 1999/00 and
2000/01 National Marine Recreational Fishing Surveys.

The harvest estimates provided by telephone-diary surveys between 1993
and 2001 are no longer considered reliable for various reasons. A
Recreational Technical Working Group concluded that these harvest
estimates should be used only with the following qualifications: a) they
may be very inaccurate; b) the 1996 and earlier surveys contain a
methodological error; and c) the 2000 and 2001 estimates are implausibly
high for many important fisheries. In response to these problems and the
cost and scale challenges associated with onsite methods, a National
Panel Survey was conducted for the first time throughout the 2011--12
fishing year. The panel survey used face-to-face interviews of a random
sample of 30 390 New Zealand households to recruit a panel of fishers
and non-fishers for a full year. The panel members were contacted
regularly about their fishing activities and harvest information
collected in standardised phone interviews. Harvest estimates for paua
are given in Table 2b (from Wynne-Jones et al 2014, using mean weights
from Hartill \& Davey 2015).

\begin{tabular}{l|r|r|r|r|r}
\hline
Stock & Fishers & Events & Number of paua & Total weight (t) & CV\\
\hline
PAU 1 & 39 & 63 & 43480 & 12.16 & 0.27\\
\hline
PAU 2 & 158 & 378 & 286182 & 81.85 & 0.15\\
\hline
PAU 3 & 35 & 67 & 60717 & 16.98 & 0.31\\
\hline
PAU 5A & 2 & 3 & 1487 & 0.42 & 0.76\\
\hline
PAU 5B & 5 & 5 & 2945 & 0.82 & 0.50\\
\hline
PAU 5D & 41 & 84 & 80290 & 22.45 & 0.30\\
\hline
PAU 7 & 19 & 41 & 50534 & 14.13 & 0.34\\
\hline
PAU total & 299 & 641 & 525635 & 148.82 & 0.11\\
\hline
\end{tabular}

A repeat of the National Panel Survey is being conducted over the
2017--18 October fishing year. Results are expected in early 2019.

\subsection{Customary fisheries}\label{customary-fisheries}

There is an important customary use of paua by Maori for food, and the
shells have been used extensively for decorations and fishing devices.
Limited data is available for reported customary landings in PAU 3;
however no information is available for current levels of customary take
for any other paua QMA. Kaitiaki are now in place in many areas and
estimates of customary harvest can be expected in the future.

\subsection{Illegal catch}\label{illegal-catch}

Current levels of illegal harvests are not known. In the past, annual
estimates of illegal harvest for some Fishstocks were provided by MFish
Compliance based on seizures. In the current paua stock assessments,
nominal illegal catches are used.

\subsection{Other sources of
mortality}\label{other-sources-of-mortality}

Paua may die from wounds caused by removal desiccation or osmotic and
temperature stress if they are bought to the surface. Sub-legal paua may
be subject to handling mortality by the fishery if they are removed from
the substrate to be measured. Further mortality may result indirectly
from being returned to unsuitable habitat or being lost to predators or
bacterial infection. Gerring (2003) observed paua (from PAU 7) with a
range of wounds in the laboratory and found that only a deep cut in the
foot caused significant mortality (40\% over 70 days). In the field this
injury reduced the ability of paua to right themselves and clamp
securely onto the reef, and consequently made them more vulnerable to
predators. The tool generally used by divers in PAU 7 is a custom made
stainless steel knife with a rounded tip and no sharp edges. This design
makes cutting the paua very unlikely (although abrasions and shell
damage may occur). Gerring (2003) estimated that in PAU 7, 37\% of paua
removed from the reef by commercial divers were undersize and were
returned to the reef. His estimate of incidental mortality associated
with fishing in PAU 7 was 0.3\% of the landed catch. Incidental fishing
mortality may be higher in areas where other types of tools and fishing
practices are used. Mortality may increase if paua are kept out of the
water for a prolonged period or returned onto sand. To date, the stock
assessments developed for paua have assumed that there is no mortality
associated with capture of undersize animals.

\section{Biology}\label{biology}

Paua are herbivores which can form large aggregations on reefs in
shallow subtidal coastal habitats. Movement is over a sufficiently small
spatial scale that the species may be considered sedentary. Paua are
broadcast spawners and spawning is thought to be annual. Habitat related
factors are an important source of variation in the post-settlement
survival of paua. Growth, morphometrics, and recruitment can vary over
short distances and may be influenced by factors such as wave exposure,
habitat structure, availability of food and population density. A
summary of generic estimates for biological parameters for paua are
presented in Table 3. Parameters specific to individual paua QMAs are
reported in the specific Working Group reports.

\begin{tabular}{l|l|l}
\hline
Fishstock & Estimate & Source\\
\hline
1. Natural mortality (M) & 0.02–0.25 & Sainsbury (1982)\\
\hline
2. Weight = a (length)b  (weight in kg, shell length in mm) & a = 2.99E—08, b = 3.303 & Schiel \& Breen (1991)\\
\hline
\end{tabular}

\section{Stocks and areas}\label{stocks-and-areas}

Using both mitochondrial and microsatellite markers Will \& Gemmell
(2008) found high levels of genetic variation within samples of H. Iris
taken from 25 locations spread throughout New Zealand. They also found
two patterns of weak but significant population genetic structure.
Firstly, H. iris individuals collected from the Chatham Islands were
found to be genetically distinct from those collected from coastal sites
around the North and South Islands. Secondly a genetic discontinuity was
found loosely associated with the Cook Strait region. Genetic
discontinuities within the Cook Strait region have previously been
identified in sea stars, mussels, limpets, and chitons and are possibly
related to contemporary and/or past oceanographic and geological
conditions of the region. This split may have some implications for
management of the paua stocks, with populations on the south of the
North Island, and the north of the South Island potentially warranting
management as separate entities; a status they already receive under the
zonation of the current fisheries regions, PAU 2 in the North Island,
and PAU 7 on the South Island.

\section{Stock assessment}\label{stock-assessment}

The dates of the most recent survey or stock assessment for each QMA are
listed in Table 4.

\textbf{Table 4: Recent survey and stock assessment information for each
paua QMA}

\begin{tabular}{l|l|l|l}
\hline
QMA & Type of survey or assessment & Date & Comments\\
\hline
PAU 1 & No surveys or assessments have been undertaken &  & \\
\hline
PAU 2 & Relative abundance estimate using standardised CPUE index based on commercial catch & 2014 & Standardised CPUE showed slight oscillation without trend between 1992 and 2001 and has remained flat from 2002 until 2014.\\
\hline
PAU 3 & Quantitative assessment using a Bayesian length based model & 2013 & For the 2013 stock assessment nine model runs were conducted. The Shellfish Working Group agreed on a base case model which estimated M within the model but fixed the growth parameters as providing a reliable estimate of the status of the stocks in PAU 3 with the caveat that the model most likely underestimated uncertainty in growth but adequately estimated uncertainty in natural mortality. The status of the stock was estimated to be 52\% B0\\
\hline
PAU 4 & CPUE Standardisation & 2016 & In February 2010 the Shellfish Working Group (SFWG) agreed that, due to the lack of data of adequate quality to use in the Bayesian length-based model, a stock assessment for PAU 4 using this model was not appropriate. In 2016 an analysis of the last 14 years of CPUE data was done. This report showed a potential decline in the fishery since the early 2000s, however the poor data quality is causing considerable uncertainty about the real trend in the fishery.\\
\hline
PAU 5A & Quantitative assessment using a Bayesian length based model & 2014 & The 2014 stock assessment was conducted over two subareas of the QMA. The SFWG was satisfied that the stock assessment for both the Southern and Northern areas was reliable based on the available data. The status of the stocks was estimated to be 41\% B0 for the Southern area and 47\% B0 for the Northern area\\
\hline
PAU 5B & Quantitative assessment using a Bayesian length based model & 2018 & The 2018 Plenary accepted this assessment as best scientific information.  The status of the stock was estimated to be 47\% B0.\\
\hline
PAU 5D & Quantitative assessment using a Bayesian length based model & 2016 & The reference case model estimated that the unfished spawning stock biomass (B0) was about 2457 t (2270 – 2672 t) and the spawning stock population in 2016 (B2016) was about 35\% (28–43\%) of B0. The model projection made for three years assuming current catch levels (which includes commercial catch at and using recruitment re-sampled from the recent model estimates, suggested that the spawning stock abundance will increase to about 38\% (28–52\%) B0 over the next three years. The projection also indicated that the probability of the spawning stock biomass being above the target (40\% B0) will increase from about 14\% in 2016 to 40\% by 2019.\\
\hline
PAU 6 & Biomass estimate & 1996 & This fishery has a TACC of 1 t\\
\hline
PAU 7 & Quantitative assessment using a Bayesian length based model & 2015 & The SFWG agreed that the stock assessment was reliable based on the available data. Currently, spawning stock biomass is estimated to be 18\% B0 and is about as likely as not to be below the soft limit, with fishing intensity very likely to be above the overfishing threshold.\\
\hline
PAU 10 & No surveys or assessments have been undertaken &  & \\
\hline
\end{tabular}

\subsection{Estimates of fishery parameters and
abundance}\label{estimates-of-fishery-parameters-and-abundance}

For further information on fishery parameters and abundance specific to
each paua QMA refer to the specific Working Group report.

In 2014 standardised CPUE indices were constructed to assess relative
abundance in PAU 2. In QMAs where quantitative stock assessments have
been undertaken, standardised CPUE is also used as input data for the
Bayesian length-based stock assessment model. There is however a large
amount of literature on abalone which suggests that any apparent
stability in CPUE should be interpreted with caution and CPUE may not be
proportional to abundance as it is possible to maintain high catch rates
despite a falling biomass. This occurs because paua tend to aggregate
and, in order to maximise their catch rates, divers move from areas that
have been depleted of paua, to areas with higher density. The
consequence of this fishing behaviour is that overall abundance is
decreasing while CPUE is remaining stable. This process of
hyperstability is believed to be of less concern in PAU 3, PAU 5D and
PAU 7 because fishing in these QMAs is consistent across all fishable
areas.

In PAU 4, 5A, 5B, 5D and 7 the relative abundance of paua has also been
estimated from independent research diver surveys (RDS). In PAU 7, seven
surveys have been completed over a number of years but only two surveys
have been conducted in PAU 4. In 2009 and 2010 several reviews were
conducted (Cordue (2009) and Haist V (2010 MPI .FRR) to assess; i) the
reliability of the research diver survey index as a proxy for abundance;
and ii) whether the RDS data, when used in the paua stock assessment
models, results in model outputs that do not adequately reflect the
status of the stocks. The reviews concluded that:

\begin{itemize}
\tightlist
\item
  Due to inappropriate survey design the RDS data appear to be of very
  limited use for constructing relative abundance indices.
\item
  There was clear non-linearity in the RDS index, the form of which is
  unclear and could be potentially complex.\\
\item
  CVs of RDS index `year' effects are likely to be underestimated,
  especially at low densities.
\item
  Different abundance trends among strata reduces the reliability of RDS
  indices, and the CVs are likely not to be informative about this.
\item
  It is unlikely that the assessment model can determine the true
  non-linearity of the RDS index-abundance relationship because of the
  high variability in the RDS indices.
\item
  The non-linearity observed in the RDS indices is likely to be more
  extreme at low densities, so the RDSI is likely to mask trends when it
  is most critical to observe them.
\item
  Existing RDS data is likely to be most useful at the research stratum
  level.
\end{itemize}

\subsection{Biomass estimates}\label{biomass-estimates}

Biomass was estimated for PAU 6 in 1996 (McShane et al 1996). However
the survey area was only from Kahurangi Point to the Heaphy River.

Biomass has been estimated, as part of the stock assessments, for PAU 4,
5A, 5B, 5D and 7 (Table 4). For further information on biomass estimates
specific to each paua QMA refer to the specific Working Group report.

\subsection{Yield Estimates and
Projections}\label{yield-estimates-and-projections}

Yield estimates and projections are estimated as part of the stock
assessment process. Both are available for PAU 3, PAU 5A, PAU 5B, PAU 5D
and PAU 7. For further information on yield estimates and projections
specific to each paua QMA refer to the specific Working Group report.

\subsection{Other factors}\label{other-factors}

In the last few years the commercial fishery have been implementing
voluntary management actions in the main QMAs. These management actions
include raising the minimum harvest size and subdividing QMAs into
smaller management areas and capping catch in the different areas and in
some QMAs, not catching the full Annual Catch Entitlement (ACE) in a
particular fishing year.

\section{Environmental and ecosystem
considerations}\label{environmental-and-ecosystem-considerations}

\subsection{Ecosystem role}\label{ecosystem-role}

Paua are eaten by a range of predators, and smaller paua are generally
more vulnerable to predation. Smaller paua are consumed by blue cod
(Carbines \& Beentjes 2003), snapper (Francis 2003), banded wrasse
(Russell 1983), spotties (McCardle 1983), triplefins (McCardle 1983) and
octopus (Andrew \& Naylor 2003). Large paua are generally well protected
by their strong shells, but are still vulnerable to rock lobsters
(McCardle 1983), the large predatory starfishes Astrostole scabra and
Coscinasterias muricata (Andrew \& Naylor 2003). Large paua are also
vulnerable to predation by eagle rays (McCardle 1983), but Ayling \& Cox
(1982) suggested that eagle rays feed almost exclusively on Cook's
turban. There are no known predators that feed exclusively on paua.

Paua feed preferentially on drift algae but at high densities they also
feed by grazing attached algae. They are not generally considered to
have a large structural impact upon algal communities but at high
densities they may reduce the abundance of algae. There are no
recognised interactions with paua abundance and the abundance or
distribution of other species, with the exception of kina which, at very
high densities, appear to exclude paua (Andrew et al 2000). Research at
D'Urville Island and on Wellington's south coast suggests that there is
some negative association between paua and kina (Andrew \& MacDiarmid
1999).

\subsection{Fish and invertebrate
bycatch}\label{fish-and-invertebrate-bycatch}

Because paua are harvested by hand gathering, incidental bycatch is
limited to epibiota attached to, or within the shell. The most common
epibiont on paua shell is non-geniculate coralline algae, which, along
with most other plants and animals which settle and grow on the shell,
such as barnacles, oysters, sponges, bryozoans, and algae, appears to
have general habitat requirements (i.e.~these organisms are not
restricted to the shells of paua). Several boring and spiral-shelled
polychaete worms are commonly found in and on the shells of paua. Most
of these are found on several shellfish species, although within New
Zealand's shellfish, the onuphid polychaete Brevibrachium maculatum has
been found only in paua shell Handley, S. (2004). This species; however,
has been reported to burrow into limestone, or attach its tube to the
holdfasts of algae (Read 2004). It is also not uncommon for paua
harvesters to collect predators of paua (mainly large predatory
starfish) while fishing and to effectively remove these from the
ecosystem. The levels of these removals are unlikely to have a
significant effect on starfish populations (nor, in fact, on the
mortality of paua caused by predation).

\subsection{Incidental catch (seabirds, mammals, and protected
fish)}\label{incidental-catch-seabirds-mammals-and-protected-fish}

There is no known bycatch of threatened, endangered, or protected
species associated with the hand gathering of paua.

\subsection{Benthic interactions}\label{benthic-interactions}

The environmental impact of paua harvesting is likely to be minimal
because paua are selectively hand gathered by free divers. Habitat
contact by divers at the time of harvest is limited to the area of paua
foot attachment, and paua are usually removed with a blunt tool to
minimise damage to the flesh. The diver's body is also seldom in full
contact with the benthos. Vessels anchoring during or after fishing have
the potential to cause damage to the reef depending on the type of
diving operation (in many cases, vessels do not anchor during fishing).
Damage from anchoring is likely to be greater in areas with fragile
species such as corals than it is on shallow temperate rocky reefs.
Corals are relatively abundant at shallow depths within Fiordland, but
there are seven areas within the sounds with significant populations of
fragile species where anchoring is prohibited.

\subsection{Other considerations}\label{other-considerations}

\subsubsection{Genetic effects}\label{genetic-effects}

Fishing, environmental changes, including those caused by climate change
or pollution, could alter the genetic composition or diversity of a
species and there is some evidence to suggest that genetic changes may
occur in response to fishing of abalones. Miller et al (2009) suggested
that, in Haliotis rubra in Tasmania, localised depletion will lead to
reduced local reproductive output which may, in turn, lead to an
increase in genetic diversity because migrant larval recruitment will
contribute more to total larval recruitment. Enhancement of paua stocks
with artificially-reared juveniles has the potential to lead to genetic
effects if inappropriate broodstocks are used.

\subsubsection{Biosecurity issues}\label{biosecurity-issues}

Undaria pinnatifida is a highly invasive opportunistic kelp which
spreads mainly via fouling on boat hulls. It can form dense stands
underwater, potentially resulting in competition for light and space
which may lead to the exclusion or displacement of native plant and
animal species. Undaria may be transported on the hulls of paua dive
tenders to unaffected areas. Bluff Harbour, for example, supports a
large population of Undaria, and is one of the main ports of departure
for fishing vessels harvesting paua in Fiordland, which appears to be
devoid of Undaria (R. Naylor, personal observation). In 2010, a small
population of Undaria was found in Sunday Cove in Breaksea Sound, and
attempts to eradicate it appear to have been successful (see
\url{http://www.biosecurity.govt.nz/pests/undaria}).

\subsubsection{Kaikoura Earthquake}\label{kaikoura-earthquake}

Research is underway to investigate the influence of the November 2016
Kaikoura earthquake on paua stocks in the area of the Kaikoura coastline
that is currently closed to harvest.

\subsubsection{Marine heatwave}\label{marine-heatwave}

The effects of warming trends and ocean acidification trends, and the
marine heatwave in NZ of up to 6°C higher temperatures experienced over
summer 2017-18 have not been explored.

\section{Status of the stocks}\label{status-of-the-stocks}

The status of paua stocks PAU 2, PAU 3, PAU 4, PAU 5A, PAU 5B, PAU 5D
and PAU 7 are given in the relevant Working Group reports.

\section{For further information}\label{for-further-information}

Andrew, N; Francis, M (Eds.) (2003) The Living Reef: the ecology of New
Zealand's Living Reef. Nelson, Craig Potton Publishing.

Andrew, N L; Breen, P A; Naylor, J R; Kendrick, T H; Gerring, P K
(2000a) Stock assessment of paua Haliotis iris in PAU 7 in 1998--99. New
Zealand Fisheries Assessment Report. 2000/49.

Andrew, N; Francis, M (2003) The living Reef. The Ecology of New
Zealand's Rocky Reefs. Craig Potton Publishing ISBN 1-877333-02-6.

Andrew, N L; MacDiarmid, A B (1999) Sea urchin fisheries and potential
interactions with a kina fishery in Fiordland. Conservation Advisory
Science Notes No. 266, Department of Conservation, Wellington.

Andrew, N; Naylor, R (2003) Paua. In (Eds) Andrew, N. \& Francis, M. The
Living reef. The ecology of New Zealand's rocky reefs. Craig Potton
Publishing, Nelson, New Zealand.

Andrew, N L; Naylor, J R; Gerring, P; Notman, P R (2000b) Fishery
independent surveys of paua (Haliotis iris) in PAU 5B and 5D. New
Zealand Fisheries Assessment Report 2000/3. 21 p.

Andrew, N L; Naylor, J R; Gerring, P (2000c) A modified timed-swim
method for paua stock assessment. New Zealand Fisheries Assessment
Report 2000/4. 23 p.

Andrew, N L; Naylor, J R; Kim, S W (2002) Fishery independent surveys of
the relative abundance and size-structure of paua (Haliotis iris) in PAU
5B and PAU 5D. New Zealand Fisheries Assessment Report. 2002/41.

Annala, J H; Sullivan, K J; O'Brien, C; Iball, S (Comps.) (1998) Report
from the Fishery Assessment Plenary, May 1998: Stock assessments and
yield estimates. 409 p. (Unpublished report held in NIWA library,
Wellington.)

Ayling, T; Cox, G J (1982) Collins' guide to sea fishes of New Zealand.
Collins. Auckland. 343 p.

Boyd, R O; Gowing, L; Reilly, J L (2004) 2000--2001 National Marine
Recreational Fishing Survey: diary results and harvest estimates. Draft
New Zealand Fisheries Assessment Report. (Unpublished report held by
Fisheries New Zealand, Wellington.)

Boyd, R O; Reilly, J L (2004) 1999/2000 National Marine Recreational
Fishing Survey: harvest estimates. Draft New Zealand Fisheries
Assessment Report 2004. (Unpublished report held by Fisheries New
Zealand, Wellington.)

Bradford, E (1998) Harvest estimates from the 1996 national recreational
fishing surveys. New Zealand Fisheries Assessment Research Document
1998/16. 27 p. (Unpublished report held by NIWA library, Wellington.)

Breen, P A; Andrew, N L; Kendrick, T H (2000b) Stock assessment of paua
(Haliotis iris) in PAU 5B and PAU 5D using a new length-based model. New
Zealand Fisheries Assessment Report. 2000/33.

Breen, P A; Andrew, N L; Kendrick, T H (2000c) The 2000 stock assessment
of paua (Haliotis iris) in PAU 5B using an improved Bayesian
length-based model. New Zealand Fisheries Assessment Report. 2000/48.

Breen, P A; Andrew, N L; Kim, S W (2001) The 2001 stock assessment of
paua Haliotis iris in PAU 7. New Zealand Fisheries Assessment Report.
2001/55.

Breen, P A; Kim, S W (2004) The 2004 stock assessment of paua Haliotis
iris in PAU 5A. New Zealand Fisheries Assessment Report. 2004/40.

Breen, P A; Kim, S W (2005) The 2005 stock assessment of paua Haliotis
iris in PAU 7. New Zealand Fisheries Assessment Report. 2005/47.

Breen, P A; Kim, S W (2007) The 2006 stock assessment of paua (Haliotis
iris) stocks PAU 5A (Fiordland) and PAU 5D (Otago). New Zealand
Fisheries Assessment Report. 2007/09.

Breen, P A; Smith, A N (2008a) The 2007 stock assessment of paua
(Haliotis iris) stock PAU 5B (Stewart Island). New Zealand Fisheries
Assessment Report. 2008/05.

Breen, P A; Smith, A N (2008b) Data used in the 2007 stock assessment
for paua (Haliotis iris) stock 5B (Stewart Island). New Zealand
Fisheries Assessment Report. 2008/06.

Carbines, G.D.; Beentjes, M.P. (2003). Relative abundance of blue cod in
Dusky Sound in 2002. New Zealand Fisheries Assessment Report 2003/37. 25
p.

Cordue, P L (2009) Analysis of PAU 5A diver survey data and PCELR catch
and effort data. SeaFic and PAUMac 5 report. 45 p. (Unpublished report
held by Fisheries New Zealand, Wellington.)

Francis, R I C C (1990) A maximum likelihood stock reduction method. New
Zealand Fisheries Assessment Research Document 1990/4. 8 p. (Unpublished
report held by NIWA library, Wellington.)

Francis, M (2003) Snapper. In: p.~186-191, Andrew, N.; Francis, M.
(eds). The living reef. The ecology of New Zealand's rocky reefs. Craig
Potton Publishing, Nelson.

Gerring, P K (2003) Incidental fishing mortality of paua (Haliotis iris)
in PAU 7. New Zealand Fisheries Assessment Report 2003/56. 13 p.

Haist, V (2010) Paua research diver survey: review of data collected and
simulation study of survey method. New Zealand Fisheries Assessment
Report 2010/38.

Hartill, B; Davey, N (2015) Mean weight estimates for recreational
fisheries in 2011--12. New Zealand Fisheries Assessment Report 2015/25.

Kendrick, T H; Andrew, N L (2000) Catch and effort statistics and a
summary of standardised CPUE indices for paua (Haliotis iris) in PAU 5A,
5B, and 5D. New Zealand Fisheries Assessment Report. 2000/47.

McCardle, I (1983) Young paua in peril. Shellfisheries newsletter 20.

McShane, P E (1992) Paua fishery assessment 1992. New Zealand Fisheries
Assessment Research Document 1992/3. 26 p. (Unpublished report held by
NIWA library, Wellington.)

McShane, P E (1996) Patch dynamics and effects of exploitation on
abalone (Haliotis iris) populations. Fisheries Research 25: 191--199.

McShane, P E; Mercer, S; Naylor, R (1993) Paua (Haliotis spp.) fishery
assessment 1993. New Zealand Fisheries Assessment Research Document
1993/6. 22 p. (Unpublished report held by NIWA library, Wellington.)

McShane, P E; Mercer, S F; Naylor, J R; Notman, P R (1994a) Paua fishery
assessment 1994. New Zealand Fisheries Assessment Research Document
1994/16. 47 p. (Unpublished report held by NIWA library, Wellington.)

McShane, P E; Mercer, S F; Naylor, J R (1994b) Spatial variation and
commercial fishing of New Zealand abalone (Haliotis iris and
H.~australis). New Zealand Journal of Marine and Freshwater Research 28:
345--355.

McShane, P E; Mercer, S F; Naylor, J R; Notman, P R (1996) Paua
(Haliotis iris) fishery assessment in PAU 5, 6, and 7. New Zealand
Fisheries Assessment Research Document. 1996/11. (Unpublished report
held by NIWA library, Wellington.)

McShane, P E; Naylor, J R (1995) Small-scale spatial variation in
growth, size at maturity, and yield- and egg-per-recruit relations in
the New~Zealand abalone Haliotis iris. New Zealand Journal of Marine and
Freshwater Research 29: 603--612.

McShane, P E; Schiel, D R; Mercer, S F; Murray, T (1994c) Morphometric
variation in Haliotis iris (Mollusca:Gastropoda): analysis of 61
populations. New Zealand Journal of Marine and Freshwater Research 28:
357--364.

Naylor, J R; Andrew, N L (2000) Determination of growth, size
composition, and fecundity of paua at Taranaki and Banks Peninsula. New
Zealand Fisheries Assessment Report. 2000/51.

Naylor, J R; Andrew, N L (2002) Determination of paua growth in PAU 2,
5A, 5B, and 5D. New Zealand Fisheries Assessment Report. 2002/34.

Naylor, J R; Andrew, N L; Kim, S W (2003) Fishery independent surveys of
the relative abundance, size-structure and growth of paua (Haliotis
iris) in PAU 4. New Zealand Fisheries Assessment Report. 2003/08.

Naylor, J R; Kim, S W (2004) Fishery independent surveys of the relative
abundance and size-structure of paua Haliotis iris in PAU 5D. New
Zealand Fisheries Assessment Report. 2004/48.

Naylor, J R; Notman, P R; Mercer, S F; Gerring, P (1998) Paua (Haliotis
iris) fishery assessment in PAU 5, 6, and 7. New Zealand Fisheries
Assessment Research Document. 1998/05. (Unpublished report held by NIWA
library, Wellington.)

Pirker, J G (1992) Growth, shell-ring deposition and mortality of paua
(Haliotis iris Martyn) in the Kaikoura region. MSc thesis, University of
Canterbury. 165 p.

Russell, B C (1983) The food and feeding habits of rocky reef fish of
north-eastern New Zealand. New Zealand Journal of Marine and Freshwater
Research, 17:2, 121--145

Sainsbury, K J (1982) Population dynamics and fishery management of the
paua, Haliotis iris. 1. Population structure, growth, reproduction and
mortality. New Zealand Journal of Marine and Freshwater Research 16:
147--161.

Schiel, D R; Breen, P A (1991) Population structure, ageing and fishing
mortality of the New Zealand abalone Haliotis iris. Fishery Bulletin 89:
681--691.

Schwarz, A.; Taylor, R.; Hewitt, J.; Phillips, N.; Shima, J.; Cole, R.;
Budd, R. (2006). Impacts of terrestrial runoff on the biodiversity of
rocky reefs. New Zealand Aquatic Environment and Biodiversity Report No
7. 109 p.

Teirney, L D; Kilner, A R; Millar, R E; Bradford, E; Bell, J D (1997)
Estimation of recreational catch from 1991/92 to 1993/94. New Zealand
Fisheries Assessment Research Document 1997/15. 43 p. (Unpublished
report held by NIWA library, Wellington.)

Will, M C; Gemmell, N J (2008) Genetic Population Structure of Black
Foot paua. New Zealand Fisheries Research Report. GEN2007A: 37 p.
(Unpublished report held by Fisheries New Zealand, Wellington.)

Wynne-Jones,J; Gray, A; Hill, L; Heinemann, A (2014) National Panel
Survey of Marine Recreational Fishers 2011--12: Harvest Estimates. New
Zealand Fisheries Assessment Report 2014/67.

% \backmatter
% \printindex

\end{document}
