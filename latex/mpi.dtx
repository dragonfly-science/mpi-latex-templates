% \iffalse meta-comment
% The MIT License (MIT)
% 
% Copyright (c) 2014 Dragonfly Science
% 
% Permission is hereby granted, free of charge, to any person obtaining a copy
% of this software and associated documentation files (the "Software"), to deal
% in the Software without restriction, including without limitation the rights
% to use, copy, modify, merge, publish, distribute, sublicense, and/or sell
% copies of the Software, and to permit persons to whom the Software is
% furnished to do so, subject to the following conditions:
% 
% The above copyright notice and this permission notice shall be included in all
% copies or substantial portions of the Software.
% 
% THE SOFTWARE IS PROVIDED "AS IS", WITHOUT WARRANTY OF ANY KIND, EXPRESS OR
% IMPLIED, INCLUDING BUT NOT LIMITED TO THE WARRANTIES OF MERCHANTABILITY,
% FITNESS FOR A PARTICULAR PURPOSE AND NONINFRINGEMENT. IN NO EVENT SHALL THE
% AUTHORS OR COPYRIGHT HOLDERS BE LIABLE FOR ANY CLAIM, DAMAGES OR OTHER
% LIABILITY, WHETHER IN AN ACTION OF CONTRACT, TORT OR OTHERWISE, ARISING FROM,
% OUT OF OR IN CONNECTION WITH THE SOFTWARE OR THE USE OR OTHER DEALINGS IN THE
% SOFTWARE.
% \fi
%
% \iffalse
%<common|mpi-aebr|mpi-far>\NeedsTeXFormat{LaTeX2e}[1999/12/01]
%
%<common>\ProvidesPackage{mpi}[2015/01/09 v0.1 Common formatting requirements for MPI reports]
%<mpi-aebr>\ProvidesClass{mpi-aebr}[2015/01/09 v0.1 MPI AEBR format]
%<mpi-far>\ProvidesClass{mpi-far}[2015/01/09 v0.1 MPI FAR format]
%<*driver>
\documentclass{ltxdoc}
\usepackage{mpi}
\usepackage[toc]{multitoc}

\usepackage{tocloft}
\setcounter{tocdepth}{2}

\EnableCrossrefs
\CodelineIndex
\RecordChanges
\begin{document}
  \DocInput{mpi.dtx}
\end{document}
%</driver>
%\fi
%
% \CheckSum{0}
%
% \CharacterTable
%  {Upper-case    \A\B\C\D\E\F\G\H\I\J\K\L\M\N\O\P\Q\R\S\T\U\V\W\X\Y\Z
%   Lower-case    \a\b\c\d\e\f\g\h\i\j\k\l\m\n\o\p\q\r\s\t\u\v\w\x\y\z
%   Digits        \0\1\2\3\4\5\6\7\8\9
%   Exclamation   \!     Double quote  \"     Hash (number) \#
%   Dollar        \$     Percent       \%     Ampersand     \&
%   Acute accent  \'     Left paren    \(     Right paren   \)
%   Asterisk      \*     Plus          \+     Comma         \,
%   Minus         \-     Point         \.     Solidus       \/
%   Colon         \:     Semicolon     \;     Less than     \<
%   Equals        \=     Greater than  \>     Question mark \?
%   Commercial at \@     Left bracket  \[     Backslash     \\
%   Right bracket \]     Circumflex    \^     Underscore    \_
%   Grave accent  \`     Left brace    \{     Vertical bar  \|
%   Right brace   \}     Tilde         \~}
%
% \changes{0.2}{2015/02/01}{Initial Version + fixes}
%
% \GetFileInfo{mpi.dtx}
%
% \title{Formatting Ministry for Primary Industries reports}
% \author{Dragonfly Science}
% \maketitle
%
% \begin{abstract}
%  This is a set of classes that are used to format \LaTeX documents according the 
%  Ministry for Primary Industries formatting requirements. There are two document
%  classes provided: ``mpi-aebr'', for formatting Aquatic Environment and Biodiversity
%  Reports, and ``mpi-far'', for formatting Fisheries Assessment Reports. Using these
%  packages allows for producing publication ready documents.
% \end{abstract}
% \clearpage
% \tableofcontents
%  \clearpage
% \section{Class and Package Options}
%
% All options are passed down to the common layer. 
%
% Since we intend on manually hiding certain options we need a command to 
% clear it from the global options list. This is taken from 
% \url{http://tex.stackexchange.com/questions/33245/disabling-the-draft-option-in-a-package}.
% The package needs to recognise the draft option, and just
% give it to ifdraft. The rest are just default behaviour. 
%
%    \begin{macrocode}
%<*common>
\def\@clearglobaloption#1{%
  \def\@tempa{#1}%
  \def\@tempb{\@gobble}%
  \@for\next:=\@classoptionslist\do
    {\ifx\next\@tempa
       \message{Cleared  option \next\space from global list}%
     \else
       \edef\@tempb{\@tempb,\next}%
     \fi}%
  \let\@classoptionslist\@tempb
  \expandafter\ifx\@tempb\@gobble
    \let\@classoptionslist\@empty
  \fi}

\DeclareOption{draft}{
  \PassOptionsToPackage{\CurrentOption}{ifdraft}
}
\ProcessOptions\relax

\DeclareOption{draft}{
  \PassOptionsToPackage{\CurrentOption}{mpi}
  \@clearglobaloption{draft}
}
\ProcessOptions\relax
%</common>
%    \end{macrocode}
%
% \section{Templates}
% Each different class is based on a different template. The following
% styles have classes implemented:
%
% \begin{itemize}
% \item Aquatic Environment and Biodiversity Report
% \item Fisheries Assessment Report
% \end{itemize}
%
% The basic structure is the same as the base latex article class
% with small changes to the theme. All of the classes
% use a commmon style to share general definitions.
% This is provided by the mpi style.
%
%    \begin{macrocode}
%<mpi-aebr|mpi-far>\LoadClass[a4paper,twoside,final,11pt]{article}
%<mpi-aebr|mpi-far>\RequirePackage{mpi}
%    \end{macrocode}
%
% \section{Page Layout}
%
% The MPI styles use margins based on existing reports.
%
%    \begin{macrocode}
%<common>\RequirePackage[margin=2.5cm]{geometry}
%    \end{macrocode}
%
% \section{Fonts}
%
% MPI reports use standard times like fonts.
%    
%    \begin{macrocode}
%<*mpi-aebr|mpi-far>
\RequirePackage{fontspec}
\setmainfont[Mapping=tex-text,
    ItalicFont     = {Times New Roman Italic},
    BoldFont       = {Times New Roman Bold},
    BoldItalicFont = {Times New Roman Bold Italic}]{Times New Roman}
\setsansfont[Mapping=tex-text,
  ItalicFont = {Arial Italic},
  BoldFont = {Arial Bold},
  BoldItalicFont = {Arial Bold Italic}]{Arial}
\setmathrm[Mapping=tex-text,
    Numbers=Monospaced,
    ItalicFont     = {Times New Roman Italic},
    BoldFont       = {Times New Roman Bold},
    BoldItalicFont = {Times New Roman Bold Italic}]{Times New Roman}
\setmathsf[Mapping=tex-text,
    Numbers=Monospaced,
    ItalicFont     = {Times New Roman Italic},
    BoldFont       = {Times New Roman Bold},
    BoldItalicFont = {Times New Roman Bold Italic}]{Times New Roman}
\setmathtt[Mapping=tex-text, 
    Numbers=Monospaced,
    ItalicFont     = {Times New Roman Italic},
    BoldFont       = {Times New Roman Bold},
    BoldItalicFont = {Times New Roman Bold Italic}]{Times New Roman}
%</mpi-aebr|mpi-far>
%    \end{macrocode}
%
% We also set the fontsize to be 11pt.
%    \begin{macrocode}
%<common>\renewcommand{\normalsize}{\fontsize{11pt}{13pt}\selectfont}
%    \end{macrocode}
%
% Paragraphs are not indented, but there is extra spacing between them. 
%    \begin{macrocode}
%<common>\setlength{\parskip}{3mm}
%<common>\setlength{\parindent}{0mm}
%    \end{macrocode}
%
% \section{Headings}
%    \begin{macrocode}
%<*mpi-aebr|mpi-far>
\RequirePackage{titlesec}
\titleformat{\section}{\normalfont\bfseries\sffamily\uppercase}{}{0em}{\thesection.\hspace{0.8em}}
\titleformat{name=\section,numberless}{\normalfont\bfseries\sffamily\uppercase}{}{0em}{}
\titleformat{\subsection}{\normalfont\bfseries\sffamily}{}{0em}{\thesubsection\hspace{0.8em}}
\titleformat{\subsubsection}{\normalfont\bfseries\sffamily}{}{0em}{\thesubsubsection\hspace{0.8em}}
\titlespacing{\section}{0pt}{\baselineskip}{0pt}
\titlespacing{\subsection}{0pt}{\baselineskip}{0pt}
\titlespacing{\subsubsection}{0pt}{\baselineskip}{0pt}
%</mpi-aebr|mpi-far>
%    \end{macrocode}
%
% \section{Header and Footer}
%
%    Page footers must contain the report name, page and publisher. 
%
%    \begin{macrocode}
%<*mpi-aebr|mpi-far>
\usepackage{fancyhdr}

\fancypagestyle{plain}{
    \fancyhf{}
    \fancyfoot[LO,RE]{\sffamily\scriptsize Ministry for Primary Industries}
    \fancyfoot[LE]{\sffamily\scriptsize\thepage\ {\normalsize$\bullet$}\ \footertitle}
    \fancyfoot[RO]{\sffamily\scriptsize\footertitle\ {\normalsize$\bullet$}\ \thepage}
    \renewcommand{\footrulewidth}{0.3pt} 
    \renewcommand{\headrulewidth}{0pt} 
  }
%</mpi-aebr|mpi-far>
%   \end{macrocode}
%
% \section{Bibliography}
%
% Since we are using biblatex with apa we need to use polyglossia to tell
% biblatex which language we are using. Actually polyglossia doesn't support 
% british directly so we need a slight hack. We also need the bibliography to be
% a proper section. We also define some helper cite commands for familiarity. 
%
%    \begin{macrocode}
%<*mpi-aebr|mpi-far>
\RequirePackage{csquotes}
\RequirePackage{polyglossia}
\RequirePackage{xpatch}
\setdefaultlanguage[variant=newzealand]{english}
\RequirePackage[style=apa,
    backend=biber,
    uniquename=false,
    uniquelist=false,
    apamaxprtauth=999]{biblatex}
\DeclareLanguageMapping{english}{english-apa}
\defbibheading{bibliography}[References]{\section{#1}}
\newcommand{\citet}{\textcite}
\newcommand{\citep}{\parencite}
\renewcommand{\cite}{\textcite}

% Use semicolons between names in the bibliography:
\renewcommand{\multinamedelim}{;\addspace}
\AtBeginBibliography{%
  \renewcommand*{\finalnamedelim}{;\addspace}
}

% Remove space beteeen initials in the names in the bibliography:
\AtBeginBibliography{%
  \renewcommand*{\bibinitdelim}{}
}

% Use a colon, rather than a comma, between the volume and the pages in the bibliography:
\AtBeginBibliography{%
  \renewcommand*{\bibpagespunct}{\addcolon\addspace}
}

% Use a comma between multiple citations:
\renewcommand*{\multicitedelim}{,\addspace}

% Have a maximum of two authors in citation commands:
% Patches https://github.com/plk/biblatex-apa/blob/v6.8/tex/latex/biblatex-apa/cbx/apa.cbx
\xpatchnameformat{labelname}{\ifciteseen}{\ifnumcomp{\value{listtotal}}{>}{2}}{}{Patch of labelname failed}

% Redefine the nameyeardelim command, to put a space between the authors and the year in citations
% in citations, updates https://github.com/plk/biblatex-apa/blob/v6.8/tex/latex/biblatex-apa/cbx/apa.cbx 
\renewcommand*{\nameyeardelim}{\space}

% Update the author macro so that there is no period after the authors, for authors without initials, in the bibliography
% Patches https://github.com/plk/biblatex-apa/blob/v6.8/tex/latex/biblatex-apa/bbx/apa.bbx
\xpatchbibmacro{author}{\newunit\newblock}{}{}{Patch of author macro failed}

% Not sure what this one is doing!
\xpatchbibmacro{date+extrayear}{%
  \printtext[parens]%
  }{%
    \setunit{\addperiod\space}%
      \printtext%
      }{}{}

% Change the order of the publisher and the location, so it is `publisher, location` rather than `location: publisher`, in the bibliography:
\xpatchbibmacro{location+publisher}{%
  \printlist[default][1-1]{location}%
  \setunit*{\addcolon\space}%
  \printlist{publisher}%
  \newunit}{%
  \printlist[default][1-1]{publisher}%
  \setunit*{\addcomma\space}%
  \printlist{location}%
  \newunit}{}{Patch of location+publisher macro failed}

% Do not italisise the title of books or collections in the bibliography:
\DeclareFieldFormat{title}{\iffieldequalstr{titleisdescription}{true}{\mkbibbrackets{#1}}{#1\isdot}}
\DeclareFieldFormat{booktitle}{#1} %Overriding definition in biblatex.def
\DeclareFieldFormat{maintitle}{#1} %Overriding definition in biblatex.def
\DeclareFieldFormat[thesis]{title}{#1}
\DeclareFieldFormat[unpublished]{title}{#1}

% Italicise the "In:" in collections in the bibliography:
\xpatchbibmacro{in}{\bibcpstring{in}\setunit{\space}}{\mkbibemph{\bibcpstring{in}}\addcolon\setunit{\space}}{}{Patch of 'in' macro failed}

% No comma after the journal title in the bibliography:
\xpatchbibmacro{journal+issuetitle}{\addcomma}{}{}{}

% Italicise the number in articles
\DeclareFieldFormat[article]{number}{\mkbibemph{\mkbibparens{\apanum{#1}}}}

% Space between the volume and the number in articles
\xpatchbibmacro{journal+issuetitle}{\printfield{number}}{\setunit{\addspace}\printfield{number}}{}{}

%</mpi-aebr|mpi-far>
%    \end{macrocode}
%
%
% \section{Title page}
%
% Each kind of document has a different kind of header. 
% Each one shall be addressed separately. Additionally,
% each title page section for a class will be split into two parts:
% the first defines new commands for entering metadata into the titlepage
% and the second defines the actual creation of the title page. 
%
% Since some of the headings require absolute positioning,
% we load the textpos package, and a number of other packages to help with the
% title pages. We also set the graphics path so we can find common images.
%
%    \begin{macrocode}
%<*common>
\RequirePackage[absolute,overlay]{textpos}
\RequirePackage{etoolbox}
\RequirePackage{xparse}
\RequirePackage{hyphenat}
\RequirePackage{graphicx}
\graphicspath{{./}{/usr/share/texlive/texmf-dist/tex/latex/mpi/}}
%</common>
%    \end{macrocode}

% \subsubsection{Meta Data}
%
% The document has two titles. The first is the main title which goes on
% the cover of the document, while the second is a short title that goes 
% in the page footer. Both are required.  
%
%    \begin{macrocode}
%<*mpi-aebr|mpi-far>
\renewcommand\title[2]{%
  \renewcommand\@title{#2}%
  \renewcommand\footertitle{#1}%
}
\newcommand{\footertitle}{\ClassError{mpi}{No Footer title defined}{Use \noexpand\title}}
%</mpi-aebr|mpi-far>
%    \end{macrocode}
%
%
% Like in a standard document |\date| is allowed. However, it is only used for
% creating citations, and as such should contain only the year. The 
% default is changed from |\today| to the current year. There
% is also a reportmonth command to set the month. 
%
%    \begin{macrocode}
%<*mpi-aebr|mpi-far>
  \renewcommand{\@date}{\the\year}
\newcommand{\reportmonth}[1]{\renewcommand{\@month}{#1}}
\newcommand{\@month}{MMM}
%</mpi-aebr|mpi-far>
%    \end{macrocode}
%  
%
% AEBR and FAR documents require ISSN, ISBN and report numbers. All of these must have 
% sensible defaults before they are finished.
%    \begin{macrocode}
%<*mpi-aebr>
\newcommand{\@reportwallpaper}{AEBR.jpg}
\newcommand{\@issn}{1179-6480}
\newcommand{\@reportseries}{New Zealand Aquatic Environment and Biodiversity Report}
%</mpi-aebr>
%<*mpi-far>
\newcommand{\@reportwallpaper}{FAR.jpg}
\newcommand{\@issn}{1179-5352}
\newcommand{\@reportseries}{New Zealand Fisheries Assessment Report}
%</mpi-far>
%<*mpi-aebr|mpi-far>
\newcommand{\issn}[1]{\renewcommand{\@issn}{#1}}
\newcommand{\isbn}[1]{\renewcommand{\@isbn}{#1}}
\newcommand{\@isbn}{XX-XXXXX-XX}
\newcommand{\@reportno}{??}
\newcommand{\reportno}[1]{\renewcommand{\@reportno}{#1}}
\newcommand{\reportseries}[1]{\renewcommand{\@reportseries}{#1}}
\newcommand{\reportwallpaper}[1]{\renewcommand{\@reportpaper}{#1}}
%</mpi-aebr|mpi-far>
%    \end{macrocode}
%
% The report also has a customisable citation note that goes at the end of the 
% citation. This defaults to the subtitle. 
%
%
% \subsubsection{Maketitle}
%
% We use the wallpaper package to support background images, and 
% redefine the maketitle command to generate the right content. 
%    \begin{macrocode}
%<common>\RequirePackage{wallpaper}
%<mpi-aebr|mpi-far>\renewcommand{\maketitle}{
%    \end{macrocode}
%
% We start by generating metadata for the pdf. We need to at least set the pdf
% author and title. 
%    \begin{macrocode}
%<*mpi-aebr|mpi-far>
  {
    \renewcommand{\and}{ and }
    \hypersetup{  
      pdfinfo={  
        Title={\footertitle},  
        Author={\@author}
      }  
    }
  }
%</mpi-aebr|mpi-far>
%    \end{macrocode}
%
% The first page is fully covered by the special title image used by AEBRs. 
%
%    \begin{macrocode}
%<*mpi-aebr|mpi-far>
\pagestyle{empty}
\clearpage
\ThisCenterWallPaper{1.0}{\@reportwallpaper}
%</mpi-aebr|mpi-far>
%    \end{macrocode}
%
% The text is overlayed into a small box over the image, such that is it over the 
% white part of the image. 
%
%    \begin{macrocode}
%<*mpi-aebr|mpi-far>
\begin{textblock*}{0.7\textwidth}(86.1mm,61.5mm)
%</mpi-aebr|mpi-far>
%    \end{macrocode}
%
% The entire titlepage is written in sans serif font. 
%
%    \begin{macrocode}
%<*mpi-aebr|mpi-far>  
  {\sffamily
%</mpi-aebr|mpi-far>
%    \end{macrocode}
%
% The title is written in large bold font. It is ragged right rather than justified
% and should avoid hyphenation. 
%
%    \begin{macrocode}
%<*mpi-aebr|mpi-far>
  {
    \raggedright 
    \fontsize{18pt}{22pt}
    \bfseries
    \selectfont
    \nohyphens{\@title}\par
  }

%</mpi-aebr|mpi-far>
%    \end{macrocode}
%
% Each report then states that is an AEBR and its number.
%
%    \begin{macrocode}
%<*mpi-aebr|mpi-far>  
  {\@reportseries} {\@reportno}
 
%</mpi-aebr|mpi-far>
%    \end{macrocode}
%
% The authors are printed in a slightly indented list. 
%
%    \begin{macrocode}
%<*mpi-aebr|mpi-far>
  {
    \renewcommand{\and}{\\}
    \hspace{3em}
    \begin{tabular}{l}
    \@author
  \end{tabular}
}
%</mpi-aebr|mpi-far>
%    \end{macrocode}
%
% This is followed by the reports ISSN and ISBN numbers.
%
%    \begin{macrocode}
%<*mpi-aebr|mpi-far>

ISSN \@issn{} (online)\\
ISBN \@isbn{} (online)

%</mpi-aebr|mpi-far>
%    \end{macrocode}
%
% Finally the report text block (and page) finishes with the date (month year).
%    \begin{macrocode}
%<*mpi-aebr|mpi-far>
\textbf{\@month{} \@date}
}
\end{textblock*}
\mbox{}
\clearpage
%</mpi-aebr|mpi-far>
%    \end{macrocode}
%
% The inside page contains a standard set of text.
%
%    \begin{macrocode}
%<*mpi-aebr|mpi-far>
{
  \setlength{\parskip}{4ex}
  \vspace*{4ex}
Requests for further copies should be directed to:


Publications Logistics Officer\\
Ministry for Primary Industries\\
PO Box 2526\\
WELLINGTON 6140

Email: brand@mpi.govt.nz\\
Telephone: 0800 00 83 33\\
Facsimile: 04-894 0300

This publication is also available on the Ministry for Primary Industries websites at:\\
\url{http://www.mpi.govt.nz/news-and-resources/publications}\\
\url{http://fs.fish.govt.nz} go to Document library/Research reports

\textbf{\copyright{} Crown Copyright - Ministry for Primary Industries}
}
\clearpage
%</mpi-aebr|mpi-far>
%    \end{macrocode}
%
% Finally we turn the header and footer off. These get turned on by the summary command.
%
%    \begin{macrocode}
%<*mpi-aebr|mpi-far>
\pagestyle{plain}
\addtocounter{page}{-2}
\pagestyle{empty}
}
%</mpi-aebr|mpi-far>
%    \end{macrocode}
%
% \section{Table of Contents}
%
% Reports requires a table of contents that is formatted in a very specific way.
%
%    \begin{macrocode}
%<mpi-aebr|mpi-far>\RequirePackage{tocloft}
%    \end{macrocode}
%
%
% Section heading are always shown in capitals
% in the contents page. This involves rebinding contents line, so that when the 
% title is used to construct a hyperref url, it doesn't break the url. This is
% used in report and AEBR.
%
%    \begin{macrocode}
%<*mpi-aebr|mpi-far>
\let\dfly@contentsline=\contentsline
\renewcommand\contentsline[4]{\dfly@contentsline{#1}{\ifstrequal{#1}{section}{\MakeUppercase{#2}}{#2}}{#3}{#4}}
%</mpi-aebr|mpi-far>
%    \end{macrocode}
%
% The table of contents should have it's own page. And since it is printed it should
% clear a double page. It should be called the TABLE OF CONTENTS. 
%    \begin{macrocode}
%<*mpi-aebr|mpi-far>
\RequirePackage{tocloft}
\renewcommand{\cfttoctitlefont}{\bfseries\sffamily}
\addto\captionsenglish{%
  \renewcommand{\contentsname}{TABLE OF CONTENTS}
} 
\g@addto@macro\@cfttocfinish{\cleardoublepage}
\tocloftpagestyle{empty}
%</mpi-aebr|mpi-far>
%    \end{macrocode}
%
% Section lines should be sans serif.
%    \begin{macrocode}
%<*mpi-aebr|mpi-far>
\g@addto@macro\cftsecpagefont{\sffamily}
\g@addto@macro\cftsecfont{\sffamily}
%</mpi-aebr|mpi-far>
%    \end{macrocode}
%
% \section{Floats}
% 
%
% Captions are all bold.
%    \begin{macrocode}
%<mpi-aebr|mpi-far>\usepackage[font={small,bf},labelsep=colon,singlelinecheck=false]{caption}
%    \end{macrocode}
%
% \section{Draft mode}
%
% Draft mode should create a watermark accross every page to make them as draft.
% It should also turn on double spacing and some line numbering for ease of editing.
% For clarity line numbers and double spacing will only start \emph{after} the title
% page.
%    \begin{macrocode}
%<*common>
\RequirePackage{setspace}
\RequirePackage{ifdraft}
\ifdraft{
\RequirePackage{draftwatermark}
\usepackage[modulo]{lineno}
\SetWatermarkColor{black!15}
\SetWatermarkFontSize{62pt}
\SetWatermarkText{\sffamily DRAFT - Not to be quoted}

\AtEndPreamble{
\g@addto@macro\maketitle{
\linenumbers
\doublespacing
}
}
}{}
%</common>
%    \end{macrocode}
%
%  \section{Additional Packages}
% The hyperref package must be the last one loaded, so we load it here.
% We also load booktabs as that is expected to be used for all tables. 
%
%    \begin{macrocode}
%<*common>
\RequirePackage{booktabs}
\RequirePackage{hyperref}
%</common>
%    \end{macrocode}
%
%    Hyperref should be used, but the links should not be coloured or have boxes around
% them. 
%    \begin{macrocode}
%<*common>
\hypersetup{
  colorlinks=true,
  urlcolor=black,
  citecolor=black,
  anchorcolor=black,
  menucolor=black,
  linkcolor=black
}
%</common>
%    \end{macrocode}
%
% \section{Self Citing}
%
% In order to automatically generate the citation using biblatex, we
% write out a generated bib file that contains an entry for the current document.
% This must happen at the end of the prelude, so that the title and so on are defined.
% Note that new lines have the be suppressed while doing this, and |\emph| commands
% need to print out the command rather than actually do emphasis. 
%
%    \begin{macrocode}
%<*mpi-aebr|mpi-far>
\RequirePackage{lastpage}
\AtBeginDocument{
{
\renewcommand{\and}{ and }
\renewcommand{\emph}[1]{\@backslashchar emph\@charlb #1\@charrb}
\renewcommand{\\}{}
\edef\text{@article\@charlb this,^^J%
options = \@charlb skipbib \@charrb,^^J%
year = \@charlb \@date \@charrb,^^J%
title = \@charlb \@title \@charrb,^^J%
author = \@charlb \@author \@charrb,^^J%
journal = \@charlb {\@reportseries} \@reportno \@charrb,^^J%
\@charrb
}
\newwrite\tempfile
\immediate\openout\tempfile=\jobname-self.bib
\immediate\write\tempfile{\text}
\immediate\closeout\tempfile
}
}
%</mpi-aebr|mpi-far>
%    \end{macrocode}
%
%
%    There is space for an executive summary, or some other
% unnumbered starting section. 
%
%    \begin{macrocode}
%<*mpi-aebr|mpi-far>
 \NewDocumentCommand{\summary}{O {Executive Summary}}{
  \pagestyle{plain}
  \setcounter{page}{1}
  \phantomsection
  \addcontentsline{toc}{section}{\hspace{1.4em}#1}
  \section*{#1}
} 
%</mpi-aebr|mpi-far>
%    \end{macrocode}
%
%
% We also need a facility to cite the paper that is currently being written. 
%    \begin{macrocode}
%<*mpi-aebr|mpi-far>
\newcommand{\citeself}{
  \begin{refsection}[\jobname-self.bib]
  \renewcommand*{\multinamedelim}{;\space}
  \renewcommand*{\finalnamedelim}{;\space}%
  \AtNextCite{\defcounter{maxnames}{99}}
  \textbf{\citename{this}[article]{author}\ (\citefield{this}{year}).\ \citefield{this}{title}.}

  \textbf{\emph{\citefield{this}{journaltitle}}. \pageref{LastPage}\ p.}
  \end{refsection}
}
%</mpi-aebr|mpi-far>
%    \end{macrocode}
% 
% \section{Appendicies}
%
%
% We require appendicies to be printed as ``Appendix X'' in the table of 
% contents. This is accomplished by using a the appendix package with the
% titletoc option. 
%    \begin{macrocode}
%<*mpi-aebr|mpi-far>
\RequirePackage[titletoc]{appendix}
%</mpi-aebr|mpi-far>
%    \end{macrocode}
%
% In the appendix we also wish to customize a number of properties. To do this
% we simply add these to the |\appendics| command.  These are:
% \begin{itemize}
%\item Section headings printed as Appendix X 
%\item Tables, equations and figures get the appendix letter before their number
%\item Counters are reset to set
% \end{itemize}
%    \begin{macrocode}
%<*mpi-aebr|mpi-far>
\g@addto@macro\appendices{
\titleformat{\section}{\normalfont\bfseries\sffamily\uppercase}{}{0em}{APPENDIX \thesection: }
\renewcommand{\theequation}{\thesection-\arabic{equation}}
\renewcommand{\thetable}{\thesection-\arabic{table}}
\renewcommand{\thefigure}{\thesection-\arabic{figure}}
\setcounter{section}{0}  % reset counter 
\setcounter{figure}{0}  % reset counter 
\setcounter{table}{0}  % reset counter 
\setcounter{equation}{0}  % reset counter 
}
%</mpi-aebr|mpi-far>
%    \end{macrocode}

% \section{Integration}
%
% We overwrite commands used in Dragonfly report templates, so that we are able to reformat documents simply by changing
% the document class
%    \begin{macrocode}
%<*common>
\newcommand{\middleimage}[2]{}
\newcommand{\rightimage}[2]{}
\newcommand{\leftimage}[2]{}
\newcommand{\singleimage}[2]{}
\newcommand{\subtitle}[1]{}
\newcommand{\titleimage}[1]{}
\newcommand{\licence}[1]{}
%</common>
%    \end{macrocode}

\endinput
